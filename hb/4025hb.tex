\documentclass{MDXHandbook}
\usepackage{fancyhdr}
\usepackage{pdfpages}
\usepackage{longtable}
\usepackage{graphicx}
\usepackage{hyperref}
\newcommand{\cwideadline}{3$^{rd}$ November 2019}
\newcommand{\cwiideadline}{5$^{th}$ January 2020}
%\newcommand{\cwiiideadline}{31$^{st}$ March 2017}
%\newcommand{\cwiiideadline}{15$^{th}$ April 2018}
\newcommand{\resitdeadline}{1$^{st}$ August 2020}
\newcommand{\deferraldeadline}{1$^{st}$ August 2020}
\newcommand{\deferraldeadlineMay}{1$^{st}$ May 2020}
\newcommand{\moduleCode}{CST4025}
\newcommand{\moduleLeader}{Dr Ian Mitchell }
\newcommand{\theauthor}{Dr Ian Mitchell }
\newcommand{\academicyear}{2019-20}
\newcommand{\email}{i.mitchell@mdx.ac.uk}
\newcommand{\moduleTitle}{Blockchain Development}
\newcommand{\office}{T108}
\newcommand{\officehours}{Autumn \& Winter Terms: Tuesdays 1515-1615hrs; and, Wednesdays 1415-1515hrs}
\newcommand{\tel}{0208-411-6014}
\newcommand{\deptName}{Computer Science}
%\newcommand{\officehours}{Friday 1100\--1300hrs Autumn Term \\ Thursday 1400\--1600hrs Winter Term}
%\newcommand{\officehours}{Autumn Term: Mondays 1300\--1500hrs \\ Winter Term: Thursdays 1400\--1600hrs

\def\bitcoinA{%
  \leavevmode
  \vtop{\offinterlineskip %\bfseries
    \setbox0=\hbox{B}%
    \setbox2=\hbox to\wd0{\hfil\hskip-.03em
    \vrule height .3ex width .15ex\hskip .08em
    \vrule height .3ex width .15ex\hfil}
    \vbox{\copy2\box0}\box2}}





%%% NOTE: Please compile this file at least twice as the title page uses exact positioning that need two compilations
\newcommand{\mcspace}{\moduleCode\ }
%%%%%%%%%%% Please include details of the module
\modulecode{\moduleCode }
\moduletitle{\moduleTitle}
\moduleleader{\moduleLeader}
\term{\academicyear}
\duration{24 weeks}
\version{1}
\dept{\deptName}
%%%%%%%%%%% This is used by MDXHandbook.cls toformat the title page

\begin{document}
\frontmatter
\maketitle

\section*{Online location of handbook (delete if module handbook is only available online)}
This handbook can also be accessed via MyLearning.

\section*{Other formats available}
This handbook is available in a large print format. If you would like a large print copy or have other requirements for the handbook, please contact the Disability Support Service (disability@mdx.ac.uk, +44 (0)20 8411 4945). 

\section*{Disclaimer}
The material in this handbook is as accurate as possible at the date of production. You will be notified of any minor changes. If there are any major changes to the module you will be consulted prior to the changes being confirmed. 

\section*{Other documents}
Your module handbook should be read and used alongside your programme handbook and the information available to all students on My Learning, including the Academic Regulations. Your programme handbook can be found on the My Learning programme page.

\newpage
\tableofcontents

\mainmatter
\newpage
\chapter{Module Introduction}
%%% Please amend as needed

{\bf It is necessary for all students of \moduleCode\  to attend 75\% of lab sessions. Students with less than 75\% attendance will fail the module and be given a grade X. This has serious implications, since the allocation of an X grade does not allow you to resit the module and since \moduleCode\  and require you to retake the module at a later date with a further financial penalty.

Naturally, there are occasions when you may not be able to attend a lab session. Usually these absences are due to unforeseen circumstance and can be quickly remedied by doing extra work to catch-up. Prolonged absence due to exceptional circumstances need to be raised with \moduleLeader immediately via email and a deferral request submitted as soon as possible to unihelpdesk.

\section{The Module Team}
\moduleStaff[role = Module Leader, room=\office, telephone = \tel, email = \email, photo=bigcrop.jpg]{\moduleLeader}


\newpage
\section{Staff Student Communication}
%%% Please amend as needed
Students may contact staff via e-mail, phone, by dropping in to staff office hours, and by making an appointment to see them outside office hours. 
Staff will contact students by e-mail, phone, the My Learning module page and via lectures and seminars.

The team may send urgent group and/or individual messages about the module to you by email, so it is important that you read your University email regularly.
All staff have office hours, it is not necessary to book an appointment during these hours.

In the first instance problems should be dealt with by talking to a member of the module team. You can give feedback on this module to the module leader, your Student Voice Leader, and through the end of module evaluation survey.

\section{Module Overview}
%%% Please amend as needed
\subsubsection{Aims}
Blockchain Technology is changing how organisations communicate and operate, with this change there is a challenge and opportunity for Blockchain developers. This module aims to convey the required knowledge underpinning blockchain technology in order to enable students to apply it to develop solutions to practical problems.


\subsubsection{Knowledge}
On completion of this module, the successful student will be able to:
\begin{itemize}
	\item Appraise blockchain types and holistically explain blockchain anatomy
	\item Analyse a domain specific problem and determine the applicability of a blockchain solution to that problem
\end{itemize}
\subsubsection{Skill}
This module will call for the successful student to demonstrate:

\begin{itemize}
	\item The design and development of effective blockchain applications
\end{itemize}

\subsubsection{Syllabus}
\begin{itemize}
	\item  Asynchronous and procedural programming pertaining to blockchain applications
    	\item  Blockchain data structures for distributed ledger systems
    	\item  Access Control Language for distributed ledger systems
    	\item  Implementing business logic for a blockchain solution 
    	\item  Blockchain and complex system development
    	\item  Permissioned Blockchain Technologies
    	\item  Consensus Engineering
    	\item  Blockchain Engineering
    	\item  Blockchain Anatomy
\end{itemize}

\subsection{UNISTATS - learning \& teaching }
\begin{itemize}
	\item 12*1 hour lectures
	\item 12*2 hour labs
\end{itemize}

The proposed number of scheduled teaching hours: {\bf 36}

Placement Activity: N/A

Independent Study: 12*9.5 hours

The proposed number of hours a student should complete independent study: 114

\section{Learning Resources}
%%% Please amend as needed
%%% Please amend as needed
The module team are here to help and support you achieve your goals. One of the key elements to successfully completing this module is engaging with all of the learning opportunities we offer as well and working with your peers to support one another. 

\section*{Participation and engagement}
%%% Please amend as needed

This module is designed as a combination of contact sessions and independent study. This means you must attend all the allocated sessions and you must work on your own outside them. Students are expected to take an active part in all learning sessions;  lectures, lab sessions, practical classes, seminars and workshops. 

Student attendance is monitored during  \emph{labs},  and any unexplained absences will be followed up via e-mail. If for any reason you are unable to attend a session you must inform the module leader.

To make the most of this module please complete the following every week
\begin{itemize}
\item Read through the notes making a note of any points you need to discuss with your tutor.
\item Complete the set activities before the next session, making a note of any points you need to discuss with your tutor.
\item Go to the module My Learning page, make use of extra material, view the podcasts, and access the activity solutions. Make a note of anything you wish to discuss with your tutor.
\item Complete further reading from the core text online.
\end{itemize}

The module team is committed to support you and your fellow students whilst you undertake this module. In order for you to get the most out of sessions you need to come prepared and ready to contribute. Please ensure that any work set by the team has been completed before workshops. After each class please review what has been covered and make a note of anything you would like clarification on. 

It is important that you are respectful and supportive to your fellow students and tutors. Adopting this approach will create a positive atmosphere within sessions and is something you can use in your professional life. 

To access some of the rooms and specialist space used for this module you will need your University ID card. Please remember that your University ID should be carried with you always.

\section*{Lateness policy}
%%% Please amend as needed

Please ensure you are on time to sessions as tutors will start sessions promptly.  Please note that if you are more than 15 minutes late you will not be permitted to join the session. 

\section*{Mobile phones}
%%% Please amend as needed

 Please have your phones on silent throughout the session and only use them in an emergency.



\section{Indicative Lecture Plan.}
\begin{table}[tbh!]
\begin{tabular}{ l l}
Week & Title \\ \hline
 1 & Introduction to Blockchain\\
 2 & Composer: Data Modelling\\
 3 & Composer: Access Control Language\\
 4 & Case Study and Formative Feedback\\
 5 & Asynchronous Programming \& Promises \\ 
 6 & Composer: Node.js I \\  
 7 & Composer: Node.js II\\ 
 8 & Consensus Engineering\\
 9 & Smart Contracts\\
 10 & Case Study and Formative Feedback\\
 11 & Case Study and Formative Feedback\\
 12 &  Presentation\\
 \end{tabular}
	\caption{Lecture Plan, these are indicative titles }\label{ta:lp}
\end{table}

\chapter{Research Ethics}

    {\it The teaching, learning, assessment and research activities undertaken in this module have been considered and are not likely to require ethical approval. } However, please seek advice if undertaking the module entails carrying out any research activities involving human participants, human data, animals/animal products, precious artefacts, materials or data systems. If you submit work that includes data gathered from or about people, this may be treated as academic misconduct and could lead to fail grade being awarded. Research ethics approval seeks to ensure all research is designed and undertaken according to certain principles of ethical research. These include: 
\begin{enumerate}
	\item Primary concern must be given to the safety, welfare and dignity of participants, researchers, colleagues, the environment and the wider community 
	\item Consideration of risks should be undertaken before research commences with the aim of minimising risks to those involved – i.e. human participants or animal subjects, colleagues, the environment and the wider community, as well as actual or potential risks to those directly or indirectly affected by the research.
	\item Informed consent should be freely given by participants, and by a trained person when collecting or analysing human tissue (details on accessing and completing online training for gaining informed consent for HTA purposes can be found below in Section 8).
	\item Respect for the privacy, confidentiality and anonymity of participants 
	\item  Consideration of the rights of people who may be vulnerable (by virtue of perceived or actual differences in their social status, ethnic origin, gender, mental capacities, or other such characteristics) who may be less competent or able to refuse to give consent to participate
	\item Researchers have a responsibility to the general public and to their profession; as such they should balance the anticipated benefits of their research against potential harm, misuse or abuse which must be avoided 
	\item Researchers must demonstrate the highest standards of ethical conduct and research integrity. They must work within the limits of their skills, training and experience, and refrain from exploitation, dishonesty, plagiarism, infringement of intellectual property rights and the fabrication of research results. They should declare any actual or potential conflicts of interest, and where necessary take steps to resolve them. 
	\item When using human tissues for research, Human Tissue Act and Human Tissue Authority (HTA) requirements must be met. Please contact the relevant designated person (DP) in your department or the HTA Designated Individual (DI) (Dr Lucy Ghali - L.Ghali@mdx.ac.uk). Further information is provided below in the section: "Human Tissue Authority Information", see 'Governance Structure" document and SOPs etc.  
	\item Research should not involve any illegal activity, and researchers must comply with all relevant laws.

		For more information about ethics go to the Middlesex Online Research Ethics (MORE) system which has information and guidance to help you meet the highest standards of ethical research using this link: \url{https://MOREform.mdx.ac.uk}. Information and further guidance on how to complete a research ethics application form (e.g., video guides and templates) can be found on the MORE MyLearning site\footnote{Middlesex University Definition of Research document can be located on this site}: \url{http://mdx.mrooms.net/enrol/index.php?id=12277}(Log in required) 
\end{enumerate}

\chapter{Assessment}
%%% Please amend as needed
\textbf{Formative assessment}: Formative assessments do not directly contribute to the overall module mark but they do provide an important opportunity to receive feedback on your learning. They provide an opportunity to evaluate and reflect on your understanding of what you have learnt. They also help your tutors identify what further support and guidance can be given to improve your grade. 

On this module you will have the following formal assessment opportunities: learning weeks 4 and 10\---11.

\textbf{Summative assessment}: Summative assessment is the assessed work that determines the overall module grade. It is the way the University verifies that students have met the learning outcomes at the appropriate level. 

There are \emph{two} assessment components in this module: \emph{Coursework 1 and Coursework 2}.


\section*{Feedback on your assignments}

Feedback will normally be provided within 15 working days of the published assessment component submission date.

\section*{Overall module grade}
%%% Please amend as needed
Each component of assessment will be marked as a percentage. To produce the overall module grade a weighted average percentage will be calculated and then converted to a 20-point grade using the scale below. 

%\newpage
%\begin{center}
%	\begin{sideways}
%		%% \TwentyPointGradeTable produces the correct table for your subject
%		%% Keys:
%		%%		subject		=	MSO or PDE (others can be coded in the .cls file if necessary)
%		%%		level			= 	UG or PG
%		%%		width		= table width (default is \textheight assuming it is rotated using the sideways environment
%		\TwentyPointGradeTable[subject = MSO, level = UG, width = .95\textheight]
%	\end{sideways}
%\end{center}

%\pagebreak
\section*{Assessment process}
\moduleCode is composed of {\it two} in-course assignments. These deadlines are correct, but may be subject to change. Table \ref{ta:cw} shows the important dates associated with each of the in-course assignments. The table also indicates that the assessment weighting, which gives the grand total of 100\%. Your final grade for \moduleCode will be calculated from adding each of these sub-totals for the in-course assignments, which must exceed 39.5\% to pass.


\begin{table}[h]
\begin{tabular}{|l|l|l|l|} \hline
Assessment 	& Formative Feedback 	& Deadline 				& \% \\ \hline\hline
Coursework 1		& Week 4 	& \cwideadline 	& 25 \\ \hline
Coursework 2		& Week 10\---11 & \cwiideadline & 75 \\ \hline
Deferred Coursework 1  & N/A & \deferraldeadline & 25\\ \hline
Deferred Coursework 2  & N/A   & \deferraldeadline & 75\\ \hline
Resit Coursework 1 & N/A & \resitdeadline & 25\\ \hline
Resit Coursework 2 & N/A & \resitdeadline & 75\\ \hline
\end{tabular}
\caption{In-course Assignments for \moduleCode}\label{ta:cw}
\end{table}


The following diagram provides an overview of the marking process for your module assessment. Details of the programme external examiner can be found in the programme handbook.

\begin{center}
\begin{tikzpicture}
\newcommand{\arrownode}[3]{
		\draw #1 ++ (2,0.75) node  [right, rectangle, draw, thick, minimum height = 1.5cm, text width = {.85\textwidth}] {$\cdot$ #2};
		\filldraw [thick, draw = black, fill = MDXCorporateRed] #1 --++ (1,-1) --++ (1,1) --++ (0,1.5) --++ (-1,-1) --++ (-1,1) -- #1;
		\draw #1 ++ (1,0.25) node {\textbf{\textcolor{white}{#3}}};
}
\foreach \i in {0,...,4} \coordinate (I\i) at (0,{-2 * \i});

\arrownode{(I0)}{You submit your assignment}{1}
\arrownode{(I1)}{The first marker grades the work and provides summative feedback; this could be completed anonymously depending on the assessment type.}{2}
\arrownode{(I2)}{A moderator or second marker reviews a sample of the work to quality assure the grades and feedback, to ensure they are accurate.}{3}
\arrownode{(I3)}{A sample of work is sent to the External Examiner to check that the grading and feedback is at the right level and in line with external subject benchmarks (this applies to levels 5, 6, and 7 only)}{4}
\arrownode{(I4)}{Your final grades are submitted to the subject assessment board.}{5}
\end{tikzpicture}
\end{center}


\chapter{Coursework 1}\label{ch:cw1}
\Large{\moduleCode\\
\moduleTitle\\
Coursework 1: Data Modelling\\
Academic Year: \academicyear \\
Deadline: \cwideadline \\
Coursework is worth 25\% of the module.}
\normalsize

\section{Instructions}
The aims of this coursework is to build a data model of a blockchain application. The coursework is to be completed:
\begin{itemize}
\item as an individual
\end{itemize}

\section{Introduction}\label{sec:intro}
You are to develop a blockchain application to a problem specification designed by you. Be aware of chosing trival problem specifications, since whilst they are easy to develop for solutions for, they do not have the complexity to exhibit the technical requirements to match some of the assessment criteria.

This part of the coursework is for data modelling and will form a the data model for coursework 2. Getting this correct is an important and incremental step.

A suggested plan for this coursework is as follows:
\begin{enumerate}
	\item Write problem specificaiton and its suitability for a blockchain application
	\item Model participants, assets and Transactions
	\item Code Participants, Assets and Transactions
	\item Get Formative Feedback and comment your code
	\item Make any modifications and submit
\end{enumerate}

\section{Problem Specification}\label{sec:intro}

The problem specification should be written in Chapter 1 of the document and contain the following sections:
\begin{enumerate}
	\item{\bf Use Case.} Create a subsection in Use Case model
	\item{\bf Description.} An explanation of no more than 250 words explaining the use case model
	\item{\bf Applicability} How applicable is the problem definition to blockchain, use the flowchart in lecture 1 and explain each stage in an itemised list.
\end{enumerate}

\section{Data Model}\label{sec:dataModel}
The Data Model should be written in Chapter 2 of the document and contain the following sections:
\begin{enumerate}
	\item Introduction
	\item Participants
	\item Assets
	\item Transactions
\end{enumerate}

\subsection{Introduction}
Include a class diagram of your problem specification. Any assumptions or explanations should be included and not exceed 100 words.


\subsection{Participants}
Based on your class diagram write the CTO code for the participants. Any assumptions or explanations should be included and not exceed 100 words.


\subsection{Assets}
Based on your class diagram write the CTO code for the assets. Any assumptions or explanations should be included and not exceed 100 words.

\subsection{Transactions}
Based on your class diagram write the CTO code for the transactions. Any assumptions or explanations should be included and not exceed 100 words.


\subsection{Comments}
All the above code should include appropriate comments.


\section{Documentation}\label{sec:doc}
All submissions are to be completed in \LaTeX\ document template available on myLearning. 


\section*{Assessment.}\label{sec:sa}
\setcounter{LTchunksize}{12}
\begin{longtable}{| p{120mm} | p{10mm} |}
\hline 
 Assessment Criteria & Mark\\ 
\hline \hline
\endfirsthead

\hline
 Assessment Criteria & Mark\\ 
\hline \hline
\endhead

\S\ref{sec:intro} Introduction
\begin{enumerate}
\item {\bf Originality:} 5\%
\item {\bf Specification \& Use Case:} 10\%
\item {\bf Applicability:} 5\%
\end{enumerate}
& 20\% \\ \hline

\S\ref{sec:dataModel} Data Model
\begin{enumerate}
\item {\bf Participants:} 20\%
\item {\bf Assets:} 20\%
\item {\bf Transactions:} 20\%
%\item {\bf Events:} 10\%
\item {\bf Comments:} 10\%
\end{enumerate}

& 70\% \\ \hline

\S\ref{sec:doc} Documentation
\begin{enumerate}
\item {\bf Template:} 2\%
\item {\bf References:} 5\%
\item {\bf Structure:} 3\%
%\item {\bf Ethics:} 5\%
\end{enumerate}
& 10\% \\ \hline \hline

{\bf Total} & 100\% \\ \hline

%
%\S\ref{sec:da} Data Acquisition
%\begin{enumerate}
%\item {\bf Accuracy} 5\%
%\item {\bf MD5 / SHA} 5\%
%\item {\bf Third Party match} 5\%
%\end{enumerate}
%& 15\% \\ \hline



\end{longtable}



\chapter{ Coursework 2}\label{ch:cw2}
\Large{\moduleCode\\
\moduleTitle\\
Coursework 2: Blockchain Application\\
Academic Year: \academicyear \\
Deadline: \cwiideadline \\
Coursework is worth 75\% of the module.}
\normalsize

\section{Instructions}
The aims of this coursework is to build a blockchain application. The coursework is to be completed:
\begin{itemize}
\item as an individual
\end{itemize}

\section{Introduction}\label{sec:intro}
You are to develop a blockchain application to a problem specification designed by you. 

A suggested plan for this coursework is as follows:
\begin{enumerate}
	\item Write Access Control in ACL
	\item Design a sequence diagram
	\item Write prototype Node.js code
	\item Complete BNA and test
	\item Get Formative Feedback and comment your code
	\item Make any modifications and submit
\end{enumerate}

\section{Data Model}\label{sec:dm}
Include a CTO listing from coursework 1. This is to provide some continuity between coursework 1 and 2, as well as provide the opportunity to show any improvements. Any improvements from the last submission should be explained and not exceed 250 words.

\section{Access Control language}\label{sec:acl}
Include a complete ACL listing for your blockchain application. 
Any assumptions or explanations should be included and not exceed 250 words.

\section{Business Logic}\label{sec:bl}
Include a complete sequence diagram for the system, this should show key components of the system and how they interact via transactions.

For each transaction include:
\begin{itemize}
	\item a subsection for the transaction, e.g., if my transaction is identified as {\tt'trade'} then the associated subsection should be named likewise
	\item Associated code
	\item Any assumptions, issues or restrictions that you have encountered. Remember sometimes the functionality can have bugs or issues that you could not resolve. These should not exceed 100 words per transaction subsection.
	\item Relate your transaction to the sequence diagram
\end{itemize}



\section{Presentation}\label{sec:pres}
The duration of each presentation should not exceed 5 minutes. Each presentation should show screenshots of the blockchain application developed above of the following:
\begin{enumerate}
	\item Use Case Diagram and a brief outline of the problem
	\item Show blockchain before transactions occur
	\item Users: show successful transactions between participants and assets
	\item Users: show unsuccessful transaction between participants and assets, this should either be because of the user does not have sufficient access privileges or the business logic prevents this from happening
	\item Show blockchain after transactions occur, highlighting where the event has updated the transaction
\end{enumerate}

\subsection{Comments}
All the above code should include appropriate comments.

\section{Documentation}\label{sec:doc}
All submissions are to be completed in \LaTeX\ document template available on myLearning. 
/
\section*{Assessment.}\label{sec:sa}
\setcounter{LTchunksize}{12}
\begin{longtable}{| p{120mm} | p{10mm} |}
\hline 
 Assessment Criteria & Mark\\ 
\hline \hline
\endfirsthead

\hline
 Assessment Criteria & Mark\\ 
\hline \hline
\endhead

\S\ref{sec:dm} Introduction
\begin{enumerate}
	\item {\bf Continuity \& Improvements:} 5\%
\end{enumerate}
& 5\% \\ \hline

\S\ref{sec:acl} Access Control 
\begin{enumerate}
	\item {\bf Order of ACL:} 10\%
	\item {\bf ACL participant access:} 10\%
	\item {\bf ACL conditions:} 10\%
	\item {\bf Comments:} 5\%
\end{enumerate}
& 35\% \\ \hline

\S\ref{sec:dm} Business Logic 
\begin{enumerate}
	\item {\bf Sequence Diagram:} 10\%
	\item {\bf Unrestricted transactions:} 10\%
	\item {\bf Restricted transactions:} 20\%
	\item {\bf Comments:} 5\%
\end{enumerate}
& 45\% \\ \hline

\S\ref{sec:dm} Presentation 
\begin{enumerate}
	\item {\bf Prepared Slides:} 2\%
	\item {\bf Slide Content:} 2\%
	\item {\bf Body Language:} 2\%
	\item {\bf Oral Content:} 2\%
\end{enumerate}
& 8\% \\ \hline


\S\ref{sec:doc} Documentation
\begin{enumerate}
\item {\bf Template:} 2\%
\item {\bf References:} 2\%
\item {\bf Structure:} 3\%
%\item {\bf Ethics:} 5\%
\end{enumerate}
& 7\% \\ \hline \hline

{\bf Total} & 100\% \\ \hline

%
%\S\ref{sec:da} Data Acquisition
%\begin{enumerate}
%\item {\bf Accuracy} 5\%
%\item {\bf MD5 / SHA} 5\%
%\item {\bf Third Party match} 5\%
%\end{enumerate}
%& 15\% \\ \hline



\end{longtable}


\chapter{Deferred Coursework}
\section{Instructions for deferrals before 1st June 2020}
If your deferral has been granted before 1$^{st}$ June 2020, then submit the coursework you deferred by \deferraldeadlineMay. 
If you deferred:
\begin{enumerate}
\item Coursework 1, then complete the Data Modelling coursework specification in Chapter \ref{ch:cw1} and submit via MyLearning by \deferraldeadlineMay
\item Coursework 2, then complete the Blockchain Application coursework in Chapter \ref{ch:cw2} and submit via MyLearning by \deferraldeadlineMay
\end{enumerate}

\section{Instruction for deferrals after 1st June 2020}
If your deferral has been granted after June then submit the coursework you deferred by \deferraldeadline.

If you deferred:
\begin{enumerate}
\item Coursework 1, then complete the Data Modelling coursework specification in Chapter \ref{ch:cw1} and submit via MyLearning by \deferraldeadline
\item Coursework 2, then complete the Blockchain Application coursework in Chapter \ref{ch:cw2} and submit via MyLearning by \deferraldeadline
\end{enumerate}


\chapter{Resit Coursework}
%\section{Instructions for resits before 1st June 2020}
%If your deferral has been granted before 1${^st}$ June 2020, then submit the coursework you deferred by \deferraldeadlineMay. 
%If you deferred:
%\begin{enumerate}
%\item Coursework 1, then complete the Data Modelling coursework specification in Chapter \ref{ch:cw1} and submit via MyLearning by \deferraldeadlineMay
%\item Coursework 2, then complete the Blockchain Application coursework in Chapter \ref{ch:cw2} and submit via MyLearning by \deferraldeadlineMay
%\end{enumerate}
%
\section{Instruction for Resits}

If you have a resit coursework code for your first attempt then, resubmit the coursework you failed:
\begin{enumerate}
\item If you failed Coursework 1, then complete the Data Modelling coursework specification in Chapter \ref{ch:cw1} and submit via MyLearning by \resitdeadline
\item If you failed Coursework 2, then complete the Blockchain Application coursework in Chapter \ref{ch:cw2} and submit via MyLearning by \resitdeadline
\end{enumerate}



\backmatter
\bibliographystyle{plain}
\nocite{*}
\bibliography{../lc/CST4025}



\end{document}
