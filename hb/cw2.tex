\Large{\moduleCode\\
\moduleTitle\\
Coursework 2: Blockchain Application\\
Academic Year: \academicyear \\
Deadline: \cwiideadline \\
Coursework is worth 75\% of the module.}
\normalsize

\section{Instructions}
The aims of this coursework is to build a blockchain application. The coursework is to be completed:
\begin{itemize}
\item as an individual
\end{itemize}

\section{Introduction}\label{sec:intro}
You are to develop a blockchain application to a problem specification designed by you. 

A suggested plan for this coursework is as follows:
\begin{enumerate}
	\item Write Access Control in ACL
	\item Design a sequence diagram
	\item Write prototype Node.js code
	\item Complete BNA and test
	\item Get Formative Feedback and comment your code
	\item Make any modifications and submit
\end{enumerate}

\section{Data Model}\label{sec:dm}
Include a CTO listing from coursework 1. This is to provide some continuity between coursework 1 and 2, as well as provide the opportunity to show any improvements. Any improvements from the last submission should be explained and not exceed 250 words.

\section{Access Control language}\label{sec:acl}
Include a complete ACL listing for your blockchain application. 
Any assumptions or explanations should be included and not exceed 250 words.

\section{Business Logic}\label{sec:bl}
Include a complete sequence diagram for the system, this should show key components of the system and how they interact via transactions.

For each transaction include:
\begin{itemize}
	\item a subsection for the transaction, e.g., if my transaction is identified as {\tt'trade'} then the associated subsection should be named likewise
	\item Associated code
	\item Any assumptions, issues or restrictions that you have encountered. Remember sometimes the functionality can have bugs or issues that you could not resolve. These should not exceed 100 words per transaction subsection.
	\item Relate your transaction to the sequence diagram
\end{itemize}



\section{Presentation}\label{sec:pres}
The duration of each presentation should not exceed 5 minutes. Each presentation should show screenshots of the blockchain application developed above of the following:
\begin{enumerate}
	\item Use Case Diagram and a brief outline of the problem
	\item Show blockchain before transactions occur
	\item Users: show successful transactions between participants and assets
	\item Users: show unsuccessful transaction between participants and assets, this should either be because of the user does not have sufficient access privileges or the business logic prevents this from happening
	\item Show blockchain after transactions occur, highlighting where the event has updated the transaction
\end{enumerate}

\subsection{Comments}
All the above code should include appropriate comments.

\section{Documentation}\label{sec:doc}
All submissions are to be completed in \LaTeX\ document template available on myLearning. 
/
\section*{Assessment.}\label{sec:sa}
\setcounter{LTchunksize}{12}
\begin{longtable}{| p{120mm} | p{10mm} |}
\hline 
 Assessment Criteria & Mark\\ 
\hline \hline
\endfirsthead

\hline
 Assessment Criteria & Mark\\ 
\hline \hline
\endhead

\S\ref{sec:dm} Introduction
\begin{enumerate}
	\item {\bf Continuity \& Improvements:} 5\%
\end{enumerate}
& 5\% \\ \hline

\S\ref{sec:acl} Access Control 
\begin{enumerate}
	\item {\bf Order of ACL:} 10\%
	\item {\bf ACL participant access:} 10\%
	\item {\bf ACL conditions:} 10\%
	\item {\bf Comments:} 5\%
\end{enumerate}
& 35\% \\ \hline

\S\ref{sec:dm} Business Logic 
\begin{enumerate}
	\item {\bf Sequence Diagram:} 10\%
	\item {\bf Unrestricted transactions:} 10\%
	\item {\bf Restricted transactions:} 20\%
	\item {\bf Comments:} 5\%
\end{enumerate}
& 45\% \\ \hline

\S\ref{sec:dm} Presentation 
\begin{enumerate}
	\item {\bf Prepared Slides:} 2\%
	\item {\bf Slide Content:} 2\%
	\item {\bf Body Language:} 2\%
	\item {\bf Oral Content:} 2\%
\end{enumerate}
& 8\% \\ \hline


\S\ref{sec:doc} Documentation
\begin{enumerate}
\item {\bf Template:} 2\%
\item {\bf References:} 2\%
\item {\bf Structure:} 3\%
%\item {\bf Ethics:} 5\%
\end{enumerate}
& 7\% \\ \hline \hline

{\bf Total} & 100\% \\ \hline

%
%\S\ref{sec:da} Data Acquisition
%\begin{enumerate}
%\item {\bf Accuracy} 5\%
%\item {\bf MD5 / SHA} 5\%
%\item {\bf Third Party match} 5\%
%\end{enumerate}
%& 15\% \\ \hline



\end{longtable}
