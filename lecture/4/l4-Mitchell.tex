\documentclass[pdf,table]{beamer}
\usepackage{graphicx,hyperref,pdfpages}
\usepackage{tikz}
\usepackage{textpos}
\usepackage{longtable}
\usepackage{listings}
\usepackage{color}
\usepackage{listings}
\usepackage{color}
\usepackage[style=numeric,backend=biber]{biblatex}
%
\usetikzlibrary{arrows}
\usetikzlibrary{positioning,chains,fit,shapes,calc}
\usetikzlibrary{mindmap}
\usetikzlibrary{shapes.multipart}
%
\addbibresource{../CST4025.bib}
\setbeamertemplate{bibliography item}{\insertbiblabel}
%
%defin colours
\definecolor{codegreen}{rgb}{0,0.6,0}
\definecolor{codegray}{rgb}{0.5,0.5,0.5}
\definecolor{codepurple}{rgb}{0.58,0,0.82}
\definecolor{backcolour}{rgb}{0.95,0.95,0.92}
\definecolor{delim}{rgb}{20,105,176}



\lstdefinelanguage{CTO}{
	keywords={abstract, asset, by, concept, default, enum, event, identified, Integer, o, participant, String, transaction },
	comment=[l]{//},
	comment=[s]{/*}{*/},
	string=[b]",
	sensitive=true,
}

\lstdefinelanguage{ACL}{
	keywords={transaction,condition,rule,description,participant,operation,resource,action,ALLOW,READ,ALL,CREATE,UPDATE,DELETE,ANY,DENY},
	comment=[l]{//},
	comment=[s]{/*}{*/},
	string=[b]",
	sensitive=true,
}

%define Javascript language
\lstdefinelanguage{JavaScript}{
keywords={typeof, new, true, false, catch, function, return, null, catch, switch, var, if, in, while, do, else, case, break},
keywordstyle=\color{blue}\bfseries,
ndkeywords={class, export, boolean, throw, implements, import, this},
ndkeywordstyle=\color{darkgray}\bfseries,
identifierstyle=\color{black},
sensitive=false,
comment=[l]{//},
morecomment=[s]{/*}{*/},
commentstyle=\color{purple}\ttfamily,
stringstyle=\color{red}\ttfamily,
morestring=[b]',
morestring=[b]"
}
%define json language
\colorlet{punct}{red!60!black}
\definecolor{background}{HTML}{EEEEEE}
\definecolor{delimiter}{RGB}{20,105,176}
\colorlet{numb}{magenta!60!black}

\lstdefinelanguage{json}{
    numbers=left,
    numberstyle=\scriptsize,
    stepnumber=1,
    numbersep=8pt,
    showstringspaces=false,
    breaklines=true,
    frame=lines,
    backgroundcolor=\color{background},
    literate=
     *{0}{{{\color{numb}0}}}{1}
      {1}{{{\color{numb}1}}}{1}
      {2}{{{\color{numb}2}}}{1}
      {3}{{{\color{numb}3}}}{1}
      {4}{{{\color{numb}4}}}{1}
      {5}{{{\color{numb}5}}}{1}
      {6}{{{\color{numb}6}}}{1}
      {7}{{{\color{numb}7}}}{1}
      {8}{{{\color{numb}8}}}{1}
      {9}{{{\color{numb}9}}}{1}
      {:}{{{\color{punct}{:}}}}{1}
      {,}{{{\color{punct}{,}}}}{1}
      {\{}{{{\color{delimiter}{\{}}}}{1}
      {\}}{{{\color{delimiter}{\}}}}}{1}
      {[}{{{\color{delimiter}{[}}}}{1}
      {]}{{{\color{delimiter}{]}}}}{1},
}
%\lstdefinelanguage{json}{
%    numbers=left,
%    numberstyle=\scriptsize,
%    stepnumber=1,
%    numbersep=8pt,
%    showstringspaces=false,
%    breaklines=true,
%    frame=lines,
%    backgroundcolor=\color{backcolour},
%    literate=
%     *{\{}{{{\color{delim}{\{}}}}{1}
%      {\}}{{{\color{delim}{\}}}}}{1}
%      {[}{{{\color{delim}{[}}}}{1}
%      {]}{{{\color{delim}{]}}}}{1},
%}



\lstdefinestyle{mys}{
	backgroundcolor=\color{backcolour},
	commentstyle=\color{codegreen},
	keywordstyle=\color{magenta},
	stringstyle=\color{codepurple},
	numberstyle=\color{codegray},
	basicstyle=\ttfamily\tiny,
	breakatwhitespace=false,
	breaklines=true
	captionpos=b,
	keepspaces=true,
	numbers=left,
	numbersep=5pt,
	showspaces=false,
	showstringspaces=false,
	showtabs=false,
	tabsize=2}
\lstset{style=mys}



\tikzset{every matrix/.style={ampersand replacement=\&,column sep=1.75cm,row sep=2cm},
		BTWMat/.style={ampersand replacement=\&, column sep=0.75cm,row sep=1cm},
		eulerMat/.style={ampersand replacement=\&,column sep=1.1cm,row sep=2cm},
		vertexHighlight/.style={circle,fill=red!80,inner sep=.1cm,text=white},
		vertex/.style={circle,fill=blue!80,inner sep=.1cm,text=white},
		bank/.style={rectangle,fill=blue!50,inner sep=0.1cm,text=black!80}
		edge/.style={--,line width=2pt},
		Dedge/.style={->,line width=2pt},
		DedgeT/.style={->,line width=1pt},
		BiEdge/.style={<->,line width=2pt},
		BiEdgeT/.style={BiEdge,line width=1pt},
		edgeHighlight/.style={--,line width=2pt,color=red},
		loop/.style={min distance=10mm, line width=2pt},
		loopT/.style={min distance=-10mm, line width=1pt},
		label/.style = { rectangle, rounded corners, draw,
		                 minimum width = 2em, fill = yellow!50,
		                 text = red, font = \tiny\bfseries },
		labelT/.style = { circle, draw, line width=1pt,
		                 minimum width = 1em, fill = yellow!50,
		                 text = red, font = \tiny\bfseries },
		every node/.style={align=center}}



\newcommand{\cwideadline}{3$^{rd}$ November 2019}
\newcommand{\cwiideadline}{5$^{th}$ January 2020}
%\newcommand{\cwiiideadline}{31$^{st}$ March 2017}
%\newcommand{\cwiiideadline}{15$^{th}$ April 2018}
\newcommand{\resitdeadline}{1$^{st}$ August 2020}
\newcommand{\deferraldeadline}{1$^{st}$ August 2020}
\newcommand{\deferraldeadlineMay}{1$^{st}$ May 2020}
\newcommand{\moduleCode}{CST4025}
\newcommand{\moduleLeader}{Dr Ian Mitchell }
\newcommand{\theauthor}{Dr Ian Mitchell }
\newcommand{\academicyear}{2019-20}
\newcommand{\email}{i.mitchell@mdx.ac.uk}
\newcommand{\moduleTitle}{Blockchain Development}
\newcommand{\office}{T108}
\newcommand{\officehours}{Autumn \& Winter Terms: Tuesdays 1515-1615hrs; and, Wednesdays 1415-1515hrs}
\newcommand{\tel}{0208-411-6014}
\newcommand{\deptName}{Computer Science}
%\newcommand{\officehours}{Friday 1100\--1300hrs Autumn Term \\ Thursday 1400\--1600hrs Winter Term}
%\newcommand{\officehours}{Autumn Term: Mondays 1300\--1500hrs \\ Winter Term: Thursdays 1400\--1600hrs

\def\bitcoinA{%
  \leavevmode
  \vtop{\offinterlineskip %\bfseries
    \setbox0=\hbox{B}%
    \setbox2=\hbox to\wd0{\hfil\hskip-.03em
    \vrule height .3ex width .15ex\hskip .08em
    \vrule height .3ex width .15ex\hfil}
    \vbox{\copy2\box0}\box2}}



%
\mode<presentation>{
\usetheme{Madrid}
\usecolortheme{beaver}
}
%
\newcommand{\theweek}{4}
\renewcommand{\theequation}{\theweek.\arabic{equation}}

\title[\moduleCode:L\theweek]{\moduleTitle \\ Week: \theweek \\ Title: Formative Feedback} 
\institute[]{\includegraphics[scale=0.25]{../../../logo/mdxSmall} \\ Middlesex University, \\Dept. of Computer Science, \\London}
\author[\email]{\moduleLeader}
\date{\today}




\begin{document}
	\begin{frame}
		\titlepage
	\end{frame}


\addtobeamertemplate{frametitle}{}{%
\begin{textblock*}{100mm}(.94\textwidth,-0.85cm)
\includegraphics[scale=0.1]{../../../logo/transparent}
\end{textblock*}}

	\begin{frame}{Lecture Aims}
		\begin{block}{Aims}
			The aims are to provide you with formative feedback to help you improve your submission the following week
		\end{block}
		\begin{block}{Structure}
			There is no structure to today's lecture or lab. It is a formative feedback week and provides an opportunity to provide generic feedback to improve your submission of coursework 1
		\end{block}
			
	\end{frame}

	\begin{frame}{Lecture Objectives}
		\begin{block}{Knowledge}
			\begin{itemize}
				\item Formative Feedback 
				\item Participants
				\item Assets
				\item Transactions
				\item Concepts
				\item Enumerators
				\item Coursework weightings
			\end{itemize}	
		\end{block}
\end{frame}


\begin{frame}{Module Structure}
\begin{table}[tbh!]
\begin{tabular}{ l l}
Week & Title \\ \hline
 1 & Introduction to Blockchain\\
 2 & Composer: Data Modelling\\
 3 & Composer: Access Control Language\\
 4 & Formative Feedback \\
 5 & Asynchronous Programming \& Promises \\ 
 6 & Composer: Node.js I \\  
 7 & Composer: Node.js II\\ 
 8 & Consensus Engineering\\
 9 & Smart Contracts\\
 10 & Feedback\\
 11 & Feedback\\
 12 &  Presentation\\
 \end{tabular}
	\caption{Lecture Plan, these are indicative titles }\label{ta:lp}
\end{table}
\end{frame}


\begin{frame}{Coursework Weighting}
\begin{table}[h]
\begin{tabular}{|l|l|l|l|} \hline
Assessment 	& Formative Feedback 	& Deadline 				& \% \\ \hline\hline
Coursework 1		& Week 4\---5 	& \cwideadline 	& 25 \\ \hline
Coursework 2		& Week 12	& \cwiideadline & 75 \\ \hline
Deferred Coursework 1  & N/A & \deferraldeadline & 25\\ \hline
Deferred Coursework 2  & N/A   & \deferraldeadline & 75\\ \hline
Resit Coursework 1 & N/A & \resitdeadline & 25\\ \hline
Resit Coursework 2 & N/A & \resitdeadline & 75\\ \hline
\end{tabular}
\caption{In-course Assignments for \moduleCode}\label{ta:cw}
\end{table}
\end{frame}


\begin{frame}[allowframebreaks]{Coursework Spec.}
	\Large{\moduleCode\\
\moduleTitle\\
Coursework 1: Data Modelling\\
Academic Year: \academicyear \\
Deadline: \cwideadline \\
Coursework is worth 25\% of the module.}
\normalsize

\section{Instructions}
The aims of this coursework is to build a data model of a blockchain application. The coursework is to be completed:
\begin{itemize}
\item as an individual
\end{itemize}

\section{Introduction}\label{sec:intro}
You are to develop a blockchain application to a problem specification designed by you. Be aware of chosing trival problem specifications, since whilst they are easy to develop for solutions for, they do not have the complexity to exhibit the technical requirements to match some of the assessment criteria.

This part of the coursework is for data modelling and will form a the data model for coursework 2. Getting this correct is an important and incremental step.

A suggested plan for this coursework is as follows:
\begin{enumerate}
	\item Write problem specificaiton and its suitability for a blockchain application
	\item Model participants, assets and Transactions
	\item Code Participants, Assets and Transactions
	\item Get Formative Feedback and comment your code
	\item Make any modifications and submit
\end{enumerate}

\section{Problem Specification}\label{sec:intro}

The problem specification should be written in Chapter 1 of the document and contain the following sections:
\begin{enumerate}
	\item{\bf Use Case.} Create a subsection in Use Case model
	\item{\bf Description.} An explanation of no more than 250 words explaining the use case model
	\item{\bf Applicability} How applicable is the problem definition to blockchain, use the flowchart in lecture 1 and explain each stage in an itemised list.
\end{enumerate}

\section{Data Model}\label{sec:dataModel}
The Data Model should be written in Chapter 2 of the document and contain the following sections:
\begin{enumerate}
	\item Introduction
	\item Participants
	\item Assets
	\item Transactions
\end{enumerate}

\subsection{Introduction}
Include a class diagram of your problem specification. Any assumptions or explanations should be included and not exceed 100 words.


\subsection{Participants}
Based on your class diagram write the CTO code for the participants. Any assumptions or explanations should be included and not exceed 100 words.


\subsection{Assets}
Based on your class diagram write the CTO code for the assets. Any assumptions or explanations should be included and not exceed 100 words.

\subsection{Transactions}
Based on your class diagram write the CTO code for the transactions. Any assumptions or explanations should be included and not exceed 100 words.


\subsection{Comments}
All the above code should include appropriate comments.


\section{Documentation}\label{sec:doc}
All submissions are to be completed in \LaTeX\ document template available on myLearning. 


\section*{Assessment.}\label{sec:sa}
\setcounter{LTchunksize}{12}
\begin{longtable}{| p{120mm} | p{10mm} |}
\hline 
 Assessment Criteria & Mark\\ 
\hline \hline
\endfirsthead

\hline
 Assessment Criteria & Mark\\ 
\hline \hline
\endhead

\S\ref{sec:intro} Introduction
\begin{enumerate}
\item {\bf Originality:} 5\%
\item {\bf Specification \& Use Case:} 10\%
\item {\bf Applicability:} 5\%
\end{enumerate}
& 20\% \\ \hline

\S\ref{sec:dataModel} Data Model
\begin{enumerate}
\item {\bf Participants:} 20\%
\item {\bf Assets:} 20\%
\item {\bf Transactions:} 20\%
%\item {\bf Events:} 10\%
\item {\bf Comments:} 10\%
\end{enumerate}

& 70\% \\ \hline

\S\ref{sec:doc} Documentation
\begin{enumerate}
\item {\bf Template:} 2\%
\item {\bf References:} 5\%
\item {\bf Structure:} 3\%
%\item {\bf Ethics:} 5\%
\end{enumerate}
& 10\% \\ \hline \hline

{\bf Total} & 100\% \\ \hline

%
%\S\ref{sec:da} Data Acquisition
%\begin{enumerate}
%\item {\bf Accuracy} 5\%
%\item {\bf MD5 / SHA} 5\%
%\item {\bf Third Party match} 5\%
%\end{enumerate}
%& 15\% \\ \hline



\end{longtable}

\end{frame}

\begin{frame}{References}
	\printbibliography
\end{frame}
	
\begin{frame}
	\frametitle{Web Resources}
	\begin{itemize}
	\item \url{http://hyperledger.org}
	\end{itemize}
\end{frame}

\end{document}
