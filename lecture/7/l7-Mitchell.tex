\documentclass[pdf,table]{beamer}
\usepackage{graphicx,hyperref,pdfpages}
\usepackage{tikz}
\usepackage{textpos}
\usepackage{longtable}
\usepackage{listings}
\usepackage{color}
\usepackage{listings}
\usepackage{color}
\usepackage{tikz-uml}
\usepackage[style=numeric,backend=biber]{biblatex}
%
\usetikzlibrary{arrows}
\usetikzlibrary{positioning,chains,fit,shapes,calc}
\usetikzlibrary{mindmap}
\usetikzlibrary{shapes.multipart}
\usetikzlibrary{decorations.text}
%
\addbibresource{../CST4025.bib}
\setbeamertemplate{bibliography item}{\insertbiblabel}
%
%defin colours
\definecolor{codegreen}{rgb}{0,0.6,0}
\definecolor{codegray}{rgb}{0.5,0.5,0.5}
\definecolor{codepurple}{rgb}{0.58,0,0.82}
\definecolor{backcolour}{rgb}{0.95,0.95,0.92}
\definecolor{delim}{rgb}{20,105,176}



\lstdefinelanguage{CTO}{
	keywords={abstract, asset, by, concept, default, enum, event, identified, Integer, o, participant, String, transaction },
	comment=[l]{//},
	comment=[s]{/*}{*/},
	string=[b]",
	sensitive=true,
}

\lstdefinelanguage{ACL}{
	keywords={transaction,condition,rule,description,participant,operation,resource,action,ALLOW,READ,ALL,CREATE,UPDATE,DELETE,ANY,DENY},
	comment=[l]{//},
	comment=[s]{/*}{*/},
	string=[b]",
	sensitive=true,
}

%define Javascript language
\lstdefinelanguage{JavaScript}{
keywords={typeof, new, true, false, catch, function, return, null, catch, switch, var, if, in, while, do, else, case, break},
keywordstyle=\color{blue}\bfseries,
ndkeywords={class, export, boolean, throw, implements, import, this},
ndkeywordstyle=\color{darkgray}\bfseries,
identifierstyle=\color{black},
sensitive=false,
comment=[l]{//},
morecomment=[s]{/*}{*/},
commentstyle=\color{purple}\ttfamily,
stringstyle=\color{red}\ttfamily,
morestring=[b]',
morestring=[b]"
}
%define json language
\colorlet{punct}{red!60!black}
\definecolor{background}{HTML}{EEEEEE}
\definecolor{delimiter}{RGB}{20,105,176}
\colorlet{numb}{magenta!60!black}

\lstdefinelanguage{json}{
    numbers=left,
    numberstyle=\scriptsize,
    stepnumber=1,
    numbersep=8pt,
    showstringspaces=false,
    breaklines=true,
    frame=lines,
    backgroundcolor=\color{background},
    literate=
     *{0}{{{\color{numb}0}}}{1}
      {1}{{{\color{numb}1}}}{1}
      {2}{{{\color{numb}2}}}{1}
      {3}{{{\color{numb}3}}}{1}
      {4}{{{\color{numb}4}}}{1}
      {5}{{{\color{numb}5}}}{1}
      {6}{{{\color{numb}6}}}{1}
      {7}{{{\color{numb}7}}}{1}
      {8}{{{\color{numb}8}}}{1}
      {9}{{{\color{numb}9}}}{1}
      {:}{{{\color{punct}{:}}}}{1}
      {,}{{{\color{punct}{,}}}}{1}
      {\{}{{{\color{delimiter}{\{}}}}{1}
      {\}}{{{\color{delimiter}{\}}}}}{1}
      {[}{{{\color{delimiter}{[}}}}{1}
      {]}{{{\color{delimiter}{]}}}}{1},
}
%\lstdefinelanguage{json}{
%    numbers=left,
%    numberstyle=\scriptsize,
%    stepnumber=1,
%    numbersep=8pt,
%    showstringspaces=false,
%    breaklines=true,
%    frame=lines,
%    backgroundcolor=\color{backcolour},
%    literate=
%     *{\{}{{{\color{delim}{\{}}}}{1}
%      {\}}{{{\color{delim}{\}}}}}{1}
%      {[}{{{\color{delim}{[}}}}{1}
%      {]}{{{\color{delim}{]}}}}{1},
%}



\lstdefinestyle{mys}{
	backgroundcolor=\color{backcolour},
	commentstyle=\color{codegreen},
	keywordstyle=\color{magenta},
	stringstyle=\color{codepurple},
	numberstyle=\color{codegray},
	basicstyle=\ttfamily\tiny,
	breakatwhitespace=false,
	breaklines=true
	captionpos=b,
	keepspaces=true,
	numbers=left,
	numbersep=5pt,
	showspaces=false,
	showstringspaces=false,
	showtabs=false,
	tabsize=2}
\lstset{style=mys}



\tikzset{every matrix/.style={ampersand replacement=\&,column sep=1.75cm,row sep=2cm},
		BTWMat/.style={ampersand replacement=\&, column sep=0.75cm,row sep=1cm},
		eulerMat/.style={ampersand replacement=\&,column sep=1.1cm,row sep=2cm},
		vertexHighlight/.style={circle,fill=red!80,inner sep=.1cm,text=white},
		vertex/.style={circle,fill=blue!80,inner sep=.1cm,text=white},
		bank/.style={rectangle,fill=blue!50,inner sep=0.1cm,text=black!80}
		edge/.style={--,line width=2pt},
		Dedge/.style={->,line width=2pt},
		DedgeT/.style={->,line width=1pt},
		BiEdge/.style={<->,line width=2pt},
		BiEdgeT/.style={BiEdge,line width=1pt},
		edgeHighlight/.style={--,line width=2pt,color=red},
		loop/.style={min distance=10mm, line width=2pt},
		loopT/.style={min distance=-10mm, line width=1pt},
		label/.style = { rectangle, rounded corners, draw,
		                 minimum width = 2em, fill = yellow!50,
		                 text = red, font = \tiny\bfseries },
		labelT/.style = { circle, draw, line width=1pt,
		                 minimum width = 1em, fill = yellow!50,
		                 text = red, font = \tiny\bfseries },
		every node/.style={align=center}}



\newcommand{\cwideadline}{3$^{rd}$ November 2019}
\newcommand{\cwiideadline}{5$^{th}$ January 2020}
%\newcommand{\cwiiideadline}{31$^{st}$ March 2017}
%\newcommand{\cwiiideadline}{15$^{th}$ April 2018}
\newcommand{\resitdeadline}{1$^{st}$ August 2020}
\newcommand{\deferraldeadline}{1$^{st}$ August 2020}
\newcommand{\deferraldeadlineMay}{1$^{st}$ May 2020}
\newcommand{\moduleCode}{CST4025}
\newcommand{\moduleLeader}{Dr Ian Mitchell }
\newcommand{\theauthor}{Dr Ian Mitchell }
\newcommand{\academicyear}{2019-20}
\newcommand{\email}{i.mitchell@mdx.ac.uk}
\newcommand{\moduleTitle}{Blockchain Development}
\newcommand{\office}{T108}
\newcommand{\officehours}{Autumn \& Winter Terms: Tuesdays 1515-1615hrs; and, Wednesdays 1415-1515hrs}
\newcommand{\tel}{0208-411-6014}
\newcommand{\deptName}{Computer Science}
%\newcommand{\officehours}{Friday 1100\--1300hrs Autumn Term \\ Thursday 1400\--1600hrs Winter Term}
%\newcommand{\officehours}{Autumn Term: Mondays 1300\--1500hrs \\ Winter Term: Thursdays 1400\--1600hrs

\def\bitcoinA{%
  \leavevmode
  \vtop{\offinterlineskip %\bfseries
    \setbox0=\hbox{B}%
    \setbox2=\hbox to\wd0{\hfil\hskip-.03em
    \vrule height .3ex width .15ex\hskip .08em
    \vrule height .3ex width .15ex\hfil}
    \vbox{\copy2\box0}\box2}}



%
\mode<presentation>{
\usetheme{Madrid}
\usecolortheme{beaver}
}
%
\newcommand{\theweek}{7}
\renewcommand{\theequation}{\theweek.\arabic{equation}}

\title[\moduleCode:L\theweek]{\moduleTitle \\ Week: \theweek \\ Title: Node.js and TX: Lists} 
\institute[]{\includegraphics[scale=0.25]{../../../logo/mdxSmall} \\ Middlesex University, \\Dept. of Computer Science, \\London}
\author[\textcopyright \email]{\moduleLeader}
\date{\today}




\begin{document}
	\begin{frame}
		\titlepage
	\end{frame}


\addtobeamertemplate{frametitle}{}{%
\begin{textblock*}{100mm}(.94\textwidth,-0.85cm)
\includegraphics[scale=0.1]{../../../logo/transparent}
\end{textblock*}}

\begin{frame}{Lecture Objectives}
		\begin{block}{Knowledge}
			\begin{itemize}
				\item Search
				\item Lists, Arrays
				\item UpdateAll
				\item Advanced JS - more promises
				\item Pizza Delivery
				\item Events
				\item Emit
			\end{itemize}	
		\end{block}
		\begin{block}{Disclaimer}
  Unless required by applicable law or agreed to in writing, software
  distributed under the License is distributed on an "AS IS" BASIS,
  WITHOUT WARRANTIES OR CONDITIONS OF ANY KIND, either express or implied.
		\end{block}
\end{frame}

\begin{frame}{Approach}
	\begin{columns}[T]
		\begin{column}{0.48\textwidth}
			\begin{block}{Mistakes}
				We can learn a lot from bad design. Sometimes it is necessary to make mistakes in order to learn.
				Here we look at implementation of Arrays of items in registries, however, care is to be taken in 
				and warning signs should be given and the reduction in data redundancy is a good thing.
				Styles of programming will also be looked at, many coders avoid the use of promises and we will look at this approach.
			\end{block}
		\end{column}
		\begin{column}{0.48\textwidth}
			{\bf Bad Examples}
			\begin{itemize}
				\item Trader example 
				\item keep tabs on trader commodities
				\item restricted view
				\item add trader
				\item remove trader
			\end{itemize}
		\end{column}
	\end{columns}	
\end{frame}

\begin{frame}{Scenario}
	\begin{columns}[T]
		\begin{column}{0.48\textwidth}
			\begin{itemize}
				\item Each trader has a list of commodities they currently own
				\item For academic purposes
				\item {\bf Consequences from last week}
				\item removing a member of staff was difficult. Why?
			\end{itemize}
		\end{column}
		\begin{column}{0.48\textwidth}
			\begin{itemize}
				\item<2-> Only the owner can sell assets
				\item<2-> if the member of staff removed had assets
				\item<3-> these assets remain locked in, no one can sell them
				\item<4-> each trader keeps a lists of the assets they own
				\item<4-> requires updating each time a commodity changes ownership, for the seller and the buyer.
				\item<5-> when a member of staff leaves a nominated member of staff is given all the assets, and then the member of staff is deleted.
			\end{itemize}
		\end{column}
	\end{columns}	
\end{frame}


\begin{frame}{Trader}{CTO}
	\begin{columns}[T]
		\begin{column}{0.58\textwidth}
			\lstinputlisting[language=CTO]{t5/models/sample.cto}
		\end{column}
		\begin{column}{0.38\textwidth}
			{\bf Difference from last week? }
			\begin{itemize}
				\item<2-> line 25 - Array
				\item<2-> Array is to represent all the commodities owned
			\end{itemize}
		\end{column}
	\end{columns}	
\end{frame}

	
\begin{frame}{Traders}{JSON}
	\begin{columns}[T]
		\begin{column}{0.44\textwidth}
			\lstinputlisting[language=json]{t5/t0227.json}
		\end{column}
		\begin{column}{0.44\textwidth}
			\lstinputlisting[language=json]{t5/t1711.json}
		\end{column}
	\end{columns}	
\end{frame}

\begin{frame}{Commodities}{JSON}
	\begin{columns}[T]
		\begin{column}{0.44\textwidth}
			\lstinputlisting[language=json,lastline=23]{t5/commodity.json}
		\end{column}
		\begin{column}{0.44\textwidth}
			\lstinputlisting[language=json,firstline=24,firstnumber=24]{t5/commodity.json}
		\end{column}
	\end{columns}	
\end{frame}

\begin{frame}{Trade Transaction}
	\begin{itemize}
		\item Check?
		\item<2-> Buyer exists?
		\item<2-> Commodity exists?
		\item<2-> Updates?
		\item<3-> Commodity ownership
		\item<3-> Trader: commoditiesOwned array
		\item<4-> Buyer: Adding to array
		\item<4-> Seller: Removing from array
	\end{itemize}
\end{frame}

\begin{frame}{Trader Transaction}{JS - Does Buyer Exist?}
	\lstinputlisting[language=JavaScript,lastline=14]{t5/lib/sample.js}
\end{frame}

\begin{frame}{Trader Transaction}{JS - Add Commodity to Buyer Array}
	\lstinputlisting[language=JavaScript,firstline=15,firstnumber=15,lastline=21]{t5/lib/sample.js}
\end{frame}

\begin{frame}{Trader Transaction}{JS - Remove Commodity to Seller Array}
	\lstinputlisting[language=JavaScript,firstline=22,firstnumber=22,lastline=29]{t5/lib/sample.js}
\end{frame}

\begin{frame}{Trader Transaction}{JS - Commodity to Ownership updated}
	\lstinputlisting[language=JavaScript,firstline=30,firstnumber=30,lastline=37]{t5/lib/sample.js}
\end{frame}

\begin{frame}{Trader Transaction}{JS - Buyer does not exist}
	\lstinputlisting[language=JavaScript,firstline=38,firstnumber=38,lastline=44]{t5/lib/sample.js}
\end{frame}

\begin{frame}{Completing the Transaction}
	\includegraphics[scale=0.45]{gx/tx-trader.png}
\end{frame}

\begin{frame}{Transaction History}
	\includegraphics[scale=0.25]{gx/post-tx-history.png}
\end{frame}

\begin{frame}{Transaction History block}
	\lstinputlisting[language=json]{t5/tx-history.json}
\end{frame}

\begin{frame}{Traders}{JSON}
	\begin{columns}[T]
		\begin{column}{0.44\textwidth}
			\lstinputlisting[language=json]{t5/post0227.json}
		\end{column}
		\begin{column}{0.44\textwidth}
			\lstinputlisting[language=json]{t5/post1711.json}
		\end{column}
	\end{columns}	
\end{frame}

\begin{frame}{Commodities}{JSON}
	\begin{columns}[T]
		\begin{column}{0.44\textwidth}
			\lstinputlisting[language=json,lastline=23]{t5/postCommodity.json}
		\end{column}
		\begin{column}{0.44\textwidth}
			\lstinputlisting[language=json,firstline=24,firstnumber=24]{t5/postCommodity.json}
		\end{column}
	\end{columns}	
\end{frame}

\begin{frame}{Alternative}
	\begin{itemize}
		\item Pizza Delivery
		\item with a promise chain
		\item using await command
		\item The burden of access is shifted. Where?
		\item<2-> The burden is shifted from JS to ACL
		\item Look at examples on github \url{https://github.com/hyperledger/composer-sample-networks}
		\item These cannot be used for the coursework.
		\item Pizza
	\end{itemize}
\end{frame}

\begin{frame}{Pizza}{Simplified Use Case}
\centering
\begin{tikzpicture}[scale=1.0]
	\umlactor[x=0,y=1,scale=0.5,name=CUST]{CUST}
	\umlactor[x=9,y=2,scale=0.5,name=REST]{REST}
	\umlactor[x=0,y=4,scale=0.5,name=PDQ]{PDQ}
	\begin{umlsystem}[x=1,fill=blue!10]{Pizza}
		\umlusecase[x=4,y=2,name=Order]{Order}
	\end{umlsystem}
	\umlassoc{CUST}{Order}
	\umlassoc{REST}{Order}
	\umlassoc{PDQ}{Order}
\end{tikzpicture}
%\caption{ Use Case for Pizza Quality Commission and delivery service}
%\lable{fi:ucd:pdq}
\end{frame}

\begin{frame}{Pizza}{CTO - Status}
	\begin{columns}[T]
		\begin{column}{0.44\textwidth}
			\lstinputlisting[language=cto,firstline=8,firstnumber=8,lastline=14]{p2/models/model.cto}
		\end{column}
		\begin{column}{0.44\textwidth}
			\begin{itemize}
				\item<2-> lifecycle of order
				\item<2-> PLACED - create order by customer
				\item<2-> PREPARED - update by pizzOutlet
				\item<2-> DISPATCHED - update by pizzaOutlet
				\item<2-> DELIVERED - update by pizzaOutlet 
			\end{itemize}
		\end{column}
	\end{columns}	
\end{frame}

\begin{frame}{Pizza}{CTO - Size \& PizzaType}
	\begin{columns}[T]
		\begin{column}{0.55\textwidth}
			\lstinputlisting[language=cto,firstline=27,firstnumber=27,lastline=40]{p2/models/model.cto}
		\end{column}
		\begin{column}{0.35\textwidth}
			\begin{itemize}
				\item Toppings
				\item Size	
				\item Pizza Type
				\item Enumerator Types
			\end{itemize}
		\end{column}
	\end{columns}	
\end{frame}

\begin{frame}{Pizza}{CTO - Address - Customer - Restaurant}
			\lstinputlisting[language=cto,firstline=41,firstnumber=42,lastline=62]{p2/models/model.cto}
%j			\lstinputlisting[language=cto,firstline=50,firstnumber=50,lastline=58]{p2/models/model.cto}
\end{frame}

\begin{frame}{Pizza}{CTO - Order }
	\begin{columns}[T]
		\begin{column}{0.35\textwidth}
			\lstinputlisting[language=cto,firstline=70,firstnumber=70,lastline=80]{p2/models/model.cto}
		\end{column}
		\begin{column}{0.55\textwidth}
			\begin{itemize}
				\item Where does ID come from?
				\item<2-> User generated, can be pseudo-random
				\item<2-> Comment on mulitple orders
				\item<2-> array of pizzaDetails
				\item<2-> TOPPING is inaccessible
				\item<2-> Usually an order has 3 things:
					\begin{enumerate}
						\item<3-> Product: Pizza, sometimes the quantity
						\item<3-> Seller: Restaurant
						\item<3-> Buyer: Customer
					\end{enumerate}
				\item<3-> STATUS: track progress
			\end{itemize}
		\end{column}
	\end{columns}	
\end{frame}

\begin{frame}{Pizza}{CTO - Transactions \& Events}
	\lstinputlisting[language=cto,firstline=89,firstnumber=89]{p2/models/model.cto}
\end{frame}

\begin{frame}
	\begin{description}
		\item[CustomerSeeSelf:] Customers can only see themselves
	\end{description}
\end{frame}

\begin{frame}{Rules}{ACL - Customer}
	\begin{columns}[T]
		\begin{column}{0.50\textwidth}
			\lstinputlisting[language=acl,firstline=8,lastline=31,firstnumber=8]{p2/permissions.acl}
		\end{column}
		\begin{column}{0.45\textwidth}
		\begin{description}
			\item[CustomerSeeSelf:] \hfill \\ Customers can only see themselves. Condition that ensures the consumer in the order is equal to the customer.
			\item[CustomerSeePizza:] \hfill \\ Customers can see the pizzas available 
		\end{description}	
	\end{column}
	\end{columns}	
\end{frame}


\begin{frame}{Rules}{ACL - Customer}
	\begin{columns}[T]
		\begin{column}{0.50\textwidth}
			\lstinputlisting[language=acl,firstline=49,firstnumber=49,lastline=62]{p2/permissions.acl}
		\end{column}
		\begin{column}{0.45\textwidth}
		\begin{description}
			\item[customerPlaceOrder:] \hfill \\ Only a customer can place an order and access transaction {\tt placeOrder}
			\item[customerReadRestaurant:] \hfill \\ Customers are permitted to read pizzaOutlet details
		\end{description}	
	\end{column}
	\end{columns}	
\end{frame}


\begin{frame}{Rules}{ACL - Restaurant}
	\begin{columns}[T]
		\begin{column}{0.50\textwidth}
			\lstinputlisting[language=acl,firstline=33,firstnumber=33,lastline=48]{p2/permissions.acl}
		\end{column}
		\begin{column}{0.45\textwidth}
		\begin{description}
			\item[restaurantSeeSelf:] \hfill \\ Restaurant can only see themselves 
			\item[restaurantSeeOrders:] \hfill \\ Restaurant can only see orders placed at their pizzaOutlet
		\end{description}	
	\end{column}
	\end{columns}	
\end{frame}


\begin{frame}{Rules}{ACL - Restaurant}
	\begin{columns}[T]
		\begin{column}{0.40\textwidth}
			\lstinputlisting[language=acl,firstline=63,firstnumber=63,lastline=84]{p2/permissions.acl}
		\end{column}
		\begin{column}{0.55\textwidth}
		\begin{description}
			\item[restaurantReadsCustomer:] \hfill \\ restaurant can read customer details
			\item[restaurantPlaceOrder:] \hfill \\ Restaurants cannot place orders, merely read and update the status of them
			\item[restaurantProcessOrder:] \hfill \\ Restaurants can process orders from status {\tt PLACED} to {\tt PREPARED} using transaction {\tt prepareOrder}
		\end{description}	
	\end{column}
	\end{columns}	
\end{frame}


\begin{frame}{Rules}{ACL - Restaurant}
	\begin{columns}[T]
		\begin{column}{0.40\textwidth}
			\lstinputlisting[language=acl,firstline=85,firstnumber=85,lastline=98]{p2/permissions.acl}
		\end{column}
		\begin{column}{0.55\textwidth}
		\begin{description}
			\item[restaurantDispatchOrder:] \hfill \\ Restaurant can process orders from status {\tt PREPARED} to {\tt DISPATCHED} using transaction {\tt restaurantDispatchOrder}
			\item[restaurantDeliverOrder:] \hfill \\ Restaurant can process orders from status {\tt DISPATCHED} to {\tt DELIVERED} using the transaction {\tt restaurantDeliverOrder}
		\end{description}	
	\end{column}
	\end{columns}	
\end{frame}

\begin{frame}{Transactions}{JS - Place Order}
	\lstinputlisting[language=JavaScript,firstline=7,firstnumber=7,lastline=24]{p2/lib/script0.js}
\end{frame}

\begin{frame}{Transactions}{JS - Prepare Order}
	\lstinputlisting[language=JavaScript,firstline=25,firstnumber=25,lastline=49]{p2/lib/script0.js}
\end{frame}

\begin{frame}{Transactions}{JS - Dispatch Order}
	\lstinputlisting[language=JavaScript,firstline=50,firstnumber=50,lastline=74]{p2/lib/script0.js}
\end{frame}

\begin{frame}{Transactions}{JS - Deliver Order}
	\lstinputlisting[language=JavaScript,firstline=75,firstnumber=75,lastline=99]{p2/lib/script0.js}
\end{frame}





%\begin{frame}{}
%	\lstinputlisting[language=,firstline=15,firstnumber=15,lastline=21]{t5/lib/sample.js}
%\end{frame}

%\begin{frame}{}
%	\includegraphics[scale=0.45]{gx/.png}
%\end{frame}


%\begin{frame}{}{}
%	\begin{columns}[T]
%		\begin{column}{0.44\textwidth}
%			\lstinputlisting[language=]{}
%		\end{column}
%		\begin{column}{0.44\textwidth}
%		\end{column}
%	\end{columns}	
%\end{frame}

%\begin{frame}{}{}
%	\begin{columns}[T]
%		\begin{column}{0.50\textwidth}
%			\lstinputlisting[]{}
%		\end{column}
%		\begin{column}{0.45\textwidth}
%		\begin{description}
%			\item[:] \hfill \\ 
%			\item[:] \hfill \\ 
%		\end{description}	
%	\end{column}
%	\end{columns}	
%\end{frame}


%\begin{frame}{Enumerator Types}
%	\begin{columns}[T]
%		\begin{column}{0.48\textwidth}
%			\begin{lstlisting}
%			\end{lstlisting}
%		\end{column}
%		\begin{column}{0.48\textwidth}
%			{\bf XXX}
%		\end{column}
%	\end{columns}	
%\end{frame}
%
%

%\begin{frame}{X}
%	\begin{columns}[T]
%		\begin{column}{0.48\textwidth}
%			\begin{itemize}
%				\item XXX 
%			\end{itemize}
%		\end{column}
%		\begin{column}{0.48\textwidth}
%			{\bf XXX}
%		\end{column}
%	\end{columns}	
%\end{frame}
\begin{frame}[allowframebreaks]{References}
	\nocite{hyperledger:1,hyperledger:2,hyperledger:gaur2018}
	\printbibliography
\end{frame}
	
\begin{frame}{Web Resources}
	\begin{itemize}
	\item \url{http://hyperledger.org}
	\item \url{https://nodejs.org}
	\item \url{https://hyperledger.github.io/composer/latest/api/runtime-factory}
	\item \url{https://developer.mozilla.org/en-US/docs/Web/JavaScript/Reference/Global_Objects/Array}
	\item \url{https://github.com/hyperledger/composer-sample-networks}
	\item \url{https://hyperledger.github.io/composer/latest/business-network/bnd-create}
	\end{itemize}
\end{frame}
\end{document}
