\documentclass[pdf,table]{beamer}
\usepackage{graphicx,hyperref,pdfpages}
\usepackage{tikz}
\usepackage{textpos}
\usepackage{longtable}
\usetikzlibrary{arrows}
\usetikzlibrary{positioning,chains,fit,shapes,calc}
\usetikzlibrary{mindmap}



\usepackage{listings}
\usepackage{color}
%\usepackage[table]{xcolor}
%\usepackage{booktabs}


\definecolor{codegreen}{rgb}{0,0.6,0}
\definecolor{codegray}{rgb}{0.5,0.5,0.5}
\definecolor{codepurple}{rgb}{0.58,0,0.82}
\definecolor{backcolour}{rgb}{0.95,0.95,0.92}

\lstdefinestyle{mys}{
	backgroundcolor=\color{backcolour},
	commentstyle=\color{codegreen},
	keywordstyle=\color{magenta},
	stringstyle=\color{codepurple},
	numberstyle=\color{codegray},
	basicstyle=\ttfamily\scriptsize,
	breakatwhitespace=false,
	breaklines=true,
	captionpos=b,
	keepspaces=true,
	numbers=left,
	numbersep=5pt,
	showspaces=false,
	showstringspaces=false,
	showtabs=false,
	tabsize=2}
\lstset{style=mys}


\tikzset{every matrix/.style={ampersand replacement=\&,column sep=1.75cm,row sep=2cm},
		BTWMat/.style={ampersand replacement=\&, column sep=0.75cm,row sep=1cm},
		eulerMat/.style={ampersand replacement=\&,column sep=1.1cm,row sep=2cm},
		vertexHighlight/.style={circle,fill=red!80,inner sep=.1cm,text=white},
		vertex/.style={circle,fill=blue!80,inner sep=.1cm,text=white},
		bank/.style={rectangle,fill=blue!50,inner sep=0.1cm,text=black!80}
		edge/.style={--,line width=2pt},
		Dedge/.style={->,line width=2pt},
		DedgeT/.style={->,line width=1pt},
		BiEdge/.style={<->,line width=2pt},
		BiEdgeT/.style={BiEdge,line width=1pt},
		edgeHighlight/.style={--,line width=2pt,color=red},
		loop/.style={min distance=10mm, line width=2pt},
		loopT/.style={min distance=-10mm, line width=1pt},
		label/.style = { rectangle, rounded corners, draw,
		                 minimum width = 2em, fill = yellow!50,
		                 text = red, font = \tiny\bfseries },
		labelT/.style = { circle, draw, line width=1pt,
		                 minimum width = 1em, fill = yellow!50,
		                 text = red, font = \tiny\bfseries },
		every node/.style={align=center}}



\mode<presentation>{
\usetheme{Madrid}
\usecolortheme{beaver}
}




\newcommand{\cwideadline}{3$^{rd}$ November 2019}
\newcommand{\cwiideadline}{5$^{th}$ January 2020}
%\newcommand{\cwiiideadline}{31$^{st}$ March 2017}
%\newcommand{\cwiiideadline}{15$^{th}$ April 2018}
\newcommand{\resitdeadline}{1$^{st}$ August 2020}
\newcommand{\deferraldeadline}{1$^{st}$ August 2020}
\newcommand{\deferraldeadlineMay}{1$^{st}$ May 2020}
\newcommand{\moduleCode}{CST4025}
\newcommand{\moduleLeader}{Dr Ian Mitchell }
\newcommand{\theauthor}{Dr Ian Mitchell }
\newcommand{\academicyear}{2019-20}
\newcommand{\email}{i.mitchell@mdx.ac.uk}
\newcommand{\moduleTitle}{Blockchain Development}
\newcommand{\office}{T108}
\newcommand{\officehours}{Autumn \& Winter Terms: Tuesdays 1515-1615hrs; and, Wednesdays 1415-1515hrs}
\newcommand{\tel}{0208-411-6014}
\newcommand{\deptName}{Computer Science}
%\newcommand{\officehours}{Friday 1100\--1300hrs Autumn Term \\ Thursday 1400\--1600hrs Winter Term}
%\newcommand{\officehours}{Autumn Term: Mondays 1300\--1500hrs \\ Winter Term: Thursdays 1400\--1600hrs

\def\bitcoinA{%
  \leavevmode
  \vtop{\offinterlineskip %\bfseries
    \setbox0=\hbox{B}%
    \setbox2=\hbox to\wd0{\hfil\hskip-.03em
    \vrule height .3ex width .15ex\hskip .08em
    \vrule height .3ex width .15ex\hfil}
    \vbox{\copy2\box0}\box2}}




\newcommand{\theweek}{1}
\renewcommand{\theequation}{\theweek.\arabic{equation}}

\title[\moduleCode:L\theweek]{\moduleTitle \\ Week: \theweek \\ Title: Blockchain} 

%[\includegraphics[scale=0.2]{../logo/mdxSmall}]
\institute[]{\includegraphics[scale=0.25]{../../../logo/mdxSmall} \\ Middlesex University, \\Dept. of Computer Science, \\London}
\author[\email]{\moduleLeader}
\date{\today}




\begin{document}
	\begin{frame}
		\titlepage
	\end{frame}


\addtobeamertemplate{frametitle}{}{%
\begin{textblock*}{100mm}(.94\textwidth,-0.85cm)
\includegraphics[scale=0.1]{../../../logo/transparent}
\end{textblock*}}

	\begin{frame}{Module Aims}
		\begin{block}{Aims}
			Since 2008 blockchain software development has dominated innovation in technology. 
			Permissionless blockchain is the idea of decentralisation and trusting each node in the network via a combination of consensus
			and cryptography algorithms. The module will investigate the programming languages that are required to implement blockchain.

			Student will be able to demonstrate theory of blockchain technology.
			Provide knowledge, via practice, to implement a permissioned blockchain. 
		\end{block}
	\end{frame}

	\begin{frame}{Module Objectives}
		\begin{block}{Knowledge}
			\begin{itemize}
				\item Implement Blockchain
				\item Limitations of Blockchain
				\item Theoretical aspects of permission and permissionless blockchain
				\item Decentralised programming aspects
				\item Server-side scripting
				\item Data structures and programming necessary to implement blockchain
			\end{itemize}	
		\end{block}
	\end{frame}

	\begin{frame}{Aims \& Objectives}
		\begin{itemize}
			\item Introduction to private Blockchain
			\item Permission Blockchain
			\item centralised vs decentralised
			\item distributed
			\item Consensus
			\item Collaboration
			\item Security
		\end{itemize}
	\end{frame}	

%\begin{frame}{Enumerator Types}
%	\begin{columns}[T]
%		\begin{column}{0.48\textwidth}
%			\begin{lstlisting}
%			\end{lstlisting}
%		\end{column}
%		\begin{column}{0.48\textwidth}
%			{\bf XXX}
%		\end{column}
%	\end{columns}	
%\end{frame}
%
%

%\begin{frame}{X}
%	\begin{columns}[T]
%		\begin{column}{0.48\textwidth}
%			\begin{itemize}
%				\item XXX 
%			\end{itemize}
%		\end{column}
%		\begin{column}{0.48\textwidth}
%			{\bf XXX}
%		\end{column}
%	\end{columns}	
%\end{frame}

%hyperledger architecture paper 1
\begin{frame}{Hyperledger Architecture \cite{hyperledger:1}}
\begin{itemize}
	\item Consensus 
	\item Smart Contract 
	\item Communication
	\item Data Store
	\item Cryptography
	\item Policy
	\item Identity
	\item API
	\item Interoperation
\end{itemize}
\end{frame}

\begin{frame}{Hyperledger Architecture \cite{hyperledger:1}}
	{\bf Consensus}
\begin{itemize}
	\item Verify the correctness of the set of transactions
	\item A block is composed of multiple transactions
	\item Concur with other nodes
	\item Which of these can be trusted?
	\item Also provides some ordering.
	\item Consensus algorithm:
		\begin{itemize} \pause
			\item Confirms the correctness of transactions in a block, according to the consensus algorithms deployed and the policies applied.
			\item Once the block is confirmed, then it enters the blockchain, so consensus algorithm has to agree on order the blocks are added
			\item Interact and complete smart contract layer 
		\end{itemize}
\end{itemize}
\end{frame}





\begin{frame}{Data}
\end{frame}


\begin{frame}{References}
    \bibliographystyle{amsalpha}
	\nocite{hyperledger:gaur2018,ferraiolo2001proposed}          
    \bibliography{../Blockchain/forensic2}	
\end{frame}
	
\begin{frame}
	\frametitle{Web Resources}
	\begin{itemize}
	\item \url{http://hyperledger.org}
	\end{itemize}
\end{frame}

\end{document}
