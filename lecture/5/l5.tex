\documentclass[pdf,table]{beamer}
\usepackage{graphicx,hyperref,pdfpages}
\usepackage{tikz}
\usepackage{textpos}
\usepackage{longtable}
\usepackage{listings}
\usepackage{color}
\usepackage{listings}
\usepackage{color}
\usepackage[style=numeric,backend=biber]{biblatex}
%
\usetikzlibrary{arrows}
\usetikzlibrary{positioning,chains,fit,shapes,calc}
\usetikzlibrary{mindmap}
\usetikzlibrary{shapes.multipart}
\usetikzlibrary{decorations.text}
%
\addbibresource{../CST4025.bib}
\setbeamertemplate{bibliography item}{\insertbiblabel}
%
%defin colours
\definecolor{codegreen}{rgb}{0,0.6,0}
\definecolor{codegray}{rgb}{0.5,0.5,0.5}
\definecolor{codepurple}{rgb}{0.58,0,0.82}
\definecolor{backcolour}{rgb}{0.95,0.95,0.92}
\definecolor{delim}{rgb}{20,105,176}



\lstdefinelanguage{CTO}{
	keywords={abstract, asset, by, concept, default, enum, event, identified, Integer, o, participant, String, transaction },
	comment=[l]{//},
	comment=[s]{/*}{*/},
	string=[b]",
	sensitive=true,
}

\lstdefinelanguage{ACL}{
	keywords={transaction,condition,rule,description,participant,operation,resource,action,ALLOW,READ,ALL,CREATE,UPDATE,DELETE,ANY,DENY},
	comment=[l]{//},
	comment=[s]{/*}{*/},
	string=[b]",
	sensitive=true,
}

%define Javascript language
\lstdefinelanguage{JavaScript}{
keywords={typeof, new, true, false, catch, function, return, null, catch, switch, var, if, in, while, do, else, case, break},
keywordstyle=\color{blue}\bfseries,
ndkeywords={class, export, boolean, throw, implements, import, this},
ndkeywordstyle=\color{darkgray}\bfseries,
identifierstyle=\color{black},
sensitive=false,
comment=[l]{//},
morecomment=[s]{/*}{*/},
commentstyle=\color{purple}\ttfamily,
stringstyle=\color{red}\ttfamily,
morestring=[b]',
morestring=[b]"
}
%define json language
\colorlet{punct}{red!60!black}
\definecolor{background}{HTML}{EEEEEE}
\definecolor{delimiter}{RGB}{20,105,176}
\colorlet{numb}{magenta!60!black}

\lstdefinelanguage{json}{
    numbers=left,
    numberstyle=\scriptsize,
    stepnumber=1,
    numbersep=8pt,
    showstringspaces=false,
    breaklines=true,
    frame=lines,
    backgroundcolor=\color{background},
    literate=
     *{0}{{{\color{numb}0}}}{1}
      {1}{{{\color{numb}1}}}{1}
      {2}{{{\color{numb}2}}}{1}
      {3}{{{\color{numb}3}}}{1}
      {4}{{{\color{numb}4}}}{1}
      {5}{{{\color{numb}5}}}{1}
      {6}{{{\color{numb}6}}}{1}
      {7}{{{\color{numb}7}}}{1}
      {8}{{{\color{numb}8}}}{1}
      {9}{{{\color{numb}9}}}{1}
      {:}{{{\color{punct}{:}}}}{1}
      {,}{{{\color{punct}{,}}}}{1}
      {\{}{{{\color{delimiter}{\{}}}}{1}
      {\}}{{{\color{delimiter}{\}}}}}{1}
      {[}{{{\color{delimiter}{[}}}}{1}
      {]}{{{\color{delimiter}{]}}}}{1},
}
%\lstdefinelanguage{json}{
%    numbers=left,
%    numberstyle=\scriptsize,
%    stepnumber=1,
%    numbersep=8pt,
%    showstringspaces=false,
%    breaklines=true,
%    frame=lines,
%    backgroundcolor=\color{backcolour},
%    literate=
%     *{\{}{{{\color{delim}{\{}}}}{1}
%      {\}}{{{\color{delim}{\}}}}}{1}
%      {[}{{{\color{delim}{[}}}}{1}
%      {]}{{{\color{delim}{]}}}}{1},
%}



\lstdefinestyle{mys}{
	backgroundcolor=\color{backcolour},
	commentstyle=\color{codegreen},
	keywordstyle=\color{magenta},
	stringstyle=\color{codepurple},
	numberstyle=\color{codegray},
	basicstyle=\ttfamily\tiny,
	breakatwhitespace=false,
	breaklines=true
	captionpos=b,
	keepspaces=true,
	numbers=left,
	numbersep=5pt,
	showspaces=false,
	showstringspaces=false,
	showtabs=false,
	tabsize=2}
\lstset{style=mys}



\tikzset{every matrix/.style={ampersand replacement=\&,column sep=1.75cm,row sep=2cm},
		BTWMat/.style={ampersand replacement=\&, column sep=0.75cm,row sep=1cm},
		eulerMat/.style={ampersand replacement=\&,column sep=1.1cm,row sep=2cm},
		vertexHighlight/.style={circle,fill=red!80,inner sep=.1cm,text=white},
		vertex/.style={circle,fill=blue!80,inner sep=.1cm,text=white},
		bank/.style={rectangle,fill=blue!50,inner sep=0.1cm,text=black!80}
		edge/.style={--,line width=2pt},
		Dedge/.style={->,line width=2pt},
		DedgeT/.style={->,line width=1pt},
		BiEdge/.style={<->,line width=2pt},
		BiEdgeT/.style={BiEdge,line width=1pt},
		edgeHighlight/.style={--,line width=2pt,color=red},
		loop/.style={min distance=10mm, line width=2pt},
		loopT/.style={min distance=-10mm, line width=1pt},
		label/.style = { rectangle, rounded corners, draw,
		                 minimum width = 2em, fill = yellow!50,
		                 text = red, font = \tiny\bfseries },
		labelT/.style = { circle, draw, line width=1pt,
		                 minimum width = 1em, fill = yellow!50,
		                 text = red, font = \tiny\bfseries },
		every node/.style={align=center}}



\newcommand{\cwideadline}{3$^{rd}$ November 2019}
\newcommand{\cwiideadline}{5$^{th}$ January 2020}
%\newcommand{\cwiiideadline}{31$^{st}$ March 2017}
%\newcommand{\cwiiideadline}{15$^{th}$ April 2018}
\newcommand{\resitdeadline}{1$^{st}$ August 2020}
\newcommand{\deferraldeadline}{1$^{st}$ August 2020}
\newcommand{\deferraldeadlineMay}{1$^{st}$ May 2020}
\newcommand{\moduleCode}{CST4025}
\newcommand{\moduleLeader}{Dr Ian Mitchell }
\newcommand{\theauthor}{Dr Ian Mitchell }
\newcommand{\academicyear}{2019-20}
\newcommand{\email}{i.mitchell@mdx.ac.uk}
\newcommand{\moduleTitle}{Blockchain Development}
\newcommand{\office}{T108}
\newcommand{\officehours}{Autumn \& Winter Terms: Tuesdays 1515-1615hrs; and, Wednesdays 1415-1515hrs}
\newcommand{\tel}{0208-411-6014}
\newcommand{\deptName}{Computer Science}
%\newcommand{\officehours}{Friday 1100\--1300hrs Autumn Term \\ Thursday 1400\--1600hrs Winter Term}
%\newcommand{\officehours}{Autumn Term: Mondays 1300\--1500hrs \\ Winter Term: Thursdays 1400\--1600hrs

\def\bitcoinA{%
  \leavevmode
  \vtop{\offinterlineskip %\bfseries
    \setbox0=\hbox{B}%
    \setbox2=\hbox to\wd0{\hfil\hskip-.03em
    \vrule height .3ex width .15ex\hskip .08em
    \vrule height .3ex width .15ex\hfil}
    \vbox{\copy2\box0}\box2}}



%
\mode<presentation>{
\usetheme{Madrid}
\usecolortheme{beaver}
}
%
\newcommand{\theweek}{5}
\renewcommand{\theequation}{\theweek.\arabic{equation}}

\title[\moduleCode:L\theweek]{\moduleTitle \\ Week: \theweek \\ Title: Node.js} 
\institute[]{\includegraphics[scale=0.25]{../../../logo/mdxSmall} \\ Middlesex University, \\Dept. of Computer Science, \\London}
\author[\textcopyright \email]{\moduleLeader}
\date{\today}




\begin{document}
	\begin{frame}
		\titlepage
	\end{frame}


\addtobeamertemplate{frametitle}{}{%
\begin{textblock*}{100mm}(.94\textwidth,-0.85cm)
\includegraphics[scale=0.1]{../../../logo/transparent}
\end{textblock*}}

\begin{frame}{Aims \& Objectives}
	\begin{itemize}
		\item Overview of Javascript
		\item Overview of Nodejs \cite{mead:2018}
		\item A/Synchronous Programming
		\item Examples
	\end{itemize}
\end{frame}	


\begin{frame}{Node.js}{An Introduction}
	\begin{columns}[T]
		\begin{column}{0.48\textwidth}
			\begin{itemize}
				\item Client-side script
					\begin{itemize}
						\item GUI
						\item Web
						\item Mobile
						\item JS, CSS, HTML
					\end{itemize}
				\item Server-side script %simplistic overview
					\begin{itemize}
						\item Web
						\item REST
						\item HTTP
						\item Ajax
						\item Messaging
						\item lang: ASP.Net, Java, PHP
						\item tools: MySQL
					\end{itemize}
			\end{itemize}
		\end{column}
		\begin{column}{0.48\textwidth}
			{\bf Middleware}
			\begin{itemize}
				\item I/O-bound
				\item Server-side have to wait
				\item input query
				\item output result
				\item all processes halted
				\item Distributed
				\item Node.js
			\end{itemize}
		\end{column}
	\end{columns}	
\end{frame}

\begin{frame}{Node.js}{Application}
	\begin{columns}[T]
		\begin{column}{0.48\textwidth}
			\begin{block}{Felix Geisend\o{"}rfer}
				``Everything runs in parallel except your code''
			\end{block}
			\begin{itemize}
				\item Events
				\item Callbacks
				\item Listening
				\item Create callback functions that get executed in response to listening to events
				\item Non-blocking %more on https://nodejs.org/en/docs/guides/blocking-vs-non-blocking/ & p34 mead
			\end{itemize}
		\end{column}
		\begin{column}{0.48\textwidth}
			\begin{tikzpicture}
				\fill[even odd rule, mymagenta] circle(2.0);
				\node at (0,0) [font =\mytextstyle, color=white, align=center]{Event Loop};
				\node at (0,3) [font =\mytextstyle, color=black, align=center]{Load Program};
				\arcarrow{85}{3}{Something to do?}
				\arcarrow{270}{357} {Wait for something}
				\arcarrow{182}{269}{Execute Callbacks}
				\arcarrow{176}{96}{Exit, unless more to do}
			\end{tikzpicture}
		\end{column}
	\end{columns}	
\end{frame}



\begin{frame}{Node.js}
	\begin{columns}[T]
		\begin{column}{0.48\textwidth}
			{\bf Single-Threaded and Highly Parallel}
			\begin{itemize}
				\item Run code %p7 wilson
			\end{itemize}
		\end{column}
		\begin{column}{0.48\textwidth}
			{\bf Backwardism}
			\begin{itemize}
				\item
			\end{itemize}
			{\bf Why?}
			\begin{itemize}
				\item Composer
				\item Asynchronous 
				\item Non-blocking
				\item Single-Threaded
				\item Event-based
			\end{itemize}
		\end{column}
	\end{columns}	
\end{frame}


\begin{frame}{Synchronous}
	\begin{columns}[T]
		\begin{column}{0.48\textwidth}
			\begin{itemize}
				\item Sequence
			\end{itemize}
			\begin{block}{Output}<2->
				Start \\ End
			\end{block}
		\end{column}
		\begin{column}{0.48\textwidth}
			{\bf Listing}
			\lstinputlisting[language=JavaScript]{lecture/5/ex1.js}
		\end{column}
	\end{columns}	
\end{frame}

\begin{frame}{Asynchronous}%p261 mead
	\begin{columns}[T]
		\begin{column}{0.38\textwidth}
			\begin{itemize}
				\item {\tt SetTimeout({\it fn, ms})}
				\item Exec. fn after ms
				\item Order $\ne$ Code
				\item Non-blocking, continue to execute program
			\end{itemize}
			\begin{block}<2->{Output}
				Start \\ End \\ Callback
			\end{block}
		\end{column}
		\begin{column}{0.58\textwidth}
			{\bf Listing}
			\lstinputlisting[language=JavaScript]{lecture/5/ex2.js}
		\end{column}
	\end{columns}	
\end{frame}

\begin{frame}{Asynchronous}%p262 mead
	\begin{columns}[T]
		\begin{column}{0.38\textwidth}
			\begin{block}<2->{Output}
				Start \\ End \\ 2nd Callback \\ 1st Callback
			\end{block}
			\begin{itemize}
				\item Again
				\item Order $\ne$ Code
				\item Non-blocking, continue to execute program
			\end{itemize}
		\end{column}
		\begin{column}{0.58\textwidth}
			{\bf Listing}
			\lstinputlisting[language=JavaScript]{lecture/5/ex3.js}
		\end{column}
	\end{columns}	
\end{frame}

\begin{frame}{Arrow Function: No Parameter}{Syntax}
	\begin{tikzpicture}
		\node at (2,6) [code] (1){()};
		\node at (4,6) [code] (2){$=>$};
		\node at (6,6) [code] (3){\{statements\}};
		\node [rectangle, below=of 1] (4){No parameters};
		\node [rectangle, below=of 2] (5){Arrow};
		\node [rectangle, below=of 3] (6){Code};
		%
		\draw[Dedge] (4) -- (1);
		\draw[Dedge] (5) -- (2);
		\draw[Dedge] (6) -- (3);
	\end{tikzpicture}
\end{frame}


\begin{frame}{Arrow Function: Single Parameter}{Syntax}
	\begin{tikzpicture}
		\node at (2,6) [code] (1){(p1)};
		\node at (4,6) [code] (2){$=>$};
		\node at (6,6) [code] (3){\{statements\}};
		\node [rectangle, below=of 1] (4){Parameter};
		\node [rectangle, below=of 2] (5){Arrow};
		\node [rectangle, below=of 3] (6){Code};
		%
		\draw[Dedge] (4) -- (1);
		\draw[Dedge] (5) -- (2);
		\draw[Dedge] (6) -- (3);
		
	\end{tikzpicture}
\end{frame}

\begin{frame}{Arrow Function: Multiple Parameter}{Syntax}
	\begin{tikzpicture}
		\node at (2,6) [code] (1){(p1,p2)};
		\node at (4,6) [code] (2){$=>$};
		\node at (6,6) [code] (3){\{statements\}};
		\node[rectangle, below=of 1] (4){Parameter};
		\node[rectangle, below=of 2] (5){Arrow};
		\node[rectangle, below=of 3] (6){Code};
		%
		\draw[Dedge] (4) -- (1);
		\draw[Dedge] (5) -- (2);
		\draw[Dedge] (6) -- (3);
	\end{tikzpicture}
\end{frame}

\begin{frame}{Arrow Functions}{Comparison}
	\begin{columns}[T]
		\begin{column}{0.48\textwidth}
			{\bf Listing without}
			\lstinputlisting[language=JavaScript]{withoutArrowFn.js}	
		\end{column}
		\begin{column}{0.48\textwidth}
			{\bf Listing with}
			\lstinputlisting[language=JavaScript]{withArrowFn.js}
		\end{column}
	\end{columns}	
\end{frame}
%https://codeburst.io/javascript-arrow-functions-for-beginners-926947fc0cdc


\begin{frame}{Arrow Functions}{Benefits}
	\begin{itemize}
		\item Shorter
		\item Bind {\tt this} lexically
		\item 
	\end{itemize}
\end{frame}
\begin{frame}{Anonymous Callback}
	\begin{columns}[T]
		\begin{column}{0.48\textwidth}
			\begin{block}<1|only@1>{Definition}
				%When an argument in your function is a function. \\
				%A function is passed as an argument.
				Passing a function as an argument
			\end{block}
			\begin{block}<2->{Output}
				result: 50 \\ result: 15
			\end{block}
			\begin{itemize}
				\item Why?
				\item Dynamic
			\end{itemize}
			
		\end{column}
		\begin{column}{0.48\textwidth}
			{\bf Example}
			\lstinputlisting[language=JavaScript]{lecture/5/callback1.js}
		\end{column}
	\end{columns}	
\end{frame}

\begin{frame}{Named Callback}
	\begin{columns}[T]
		\begin{column}{0.48\textwidth}
			\begin{block}<2->{Output}
				result: 70 \\ result: 17 \\ result: 3
			\end{block}
			\begin{itemize}
				\item Why?
				\item Dynamic
				\item trigger automatic updates
				\item {\tt setInterval(fn,ms)}
			\end{itemize}
			
		\end{column}
		\begin{column}{0.48\textwidth}
			{\bf Example}
			\lstinputlisting[language=JavaScript]{lecture/5/callback2.js}
		\end{column}
	\end{columns}	
\end{frame}



\begin{frame}{Promises \cite{liskov:88}}
	\begin{block}{Definition \cite{parker:2015}}
		A promise is an object that serves as a placeholder for a value. That value is usually the result of an async[hronous] operation....
		When an async function is called it can immediately return a promise object. Using that object, you can register callbacks that will run when the operation succeeds or an error occurs.
	\end{block}
\end{frame}


\begin{frame}{Promise States \cite{parker:2015}}
	\begin{block}{Definitions}
		\begin{description}
			\item[Pending:] The operation has not begun or is in progress.
			\item[Fulfilled:] The operation has completed.
			\item[Rejected:] The operation could not be completed.
		\end{description}
	\end{block}
\end{frame}

\begin{frame}{Promise States Relationships\cite{parker:2015}}
	\begin{tikzpicture}
		\node  at (2,2) [zero] (1){};
		\node  at (4,2) [vertex] (2){Pending};
		\node at (8,4)  [vertex] (3){Fulfilled};
		\node at (8,0)  [vertex] (4){Rejected};
		%edges
		\draw[Dedge] (1) -- (2);
		\draw[Dedge] (2) -- (3);
		\draw[Dedge] (2) -- (4);
	\end{tikzpicture}
	%Analogy:
	% Girl: Mum, can I have some sweets
	% Mum: Be a good ri
\end{frame}

\begin{frame}{Promise Analogy}
	\begin{description}
		\item[Child:] Please can I have some sweets?
		\item[Parent:] I will give you some when you complete your homework
		\item[Promise Pending]
		\item[\vdots]
		\item<2|only@2>[Child:] Can I have some sweets now!
		\item<2|only@2>[Parent:] Have you completed your homework?
		\item<2|only@2>[Child:] No.
		\item<2|only@2>[Parent:] Then you cannot have any sweets.
		\item<2|only@2>[Promise Rejected]
		\item<3|only@3>[Child:] Can I have some sweets now?
		\item<3|only@3>[Parent:] Have you completed your homework?
		\item<3|only@3>[Child:] Yes.
		\item<3|only@3>[Parent:] Well done, I will go and get you some.
		\item<3|only@3>[Promise Pending]
		\item<3|only@3>[Parent:] They are on the table.
		\item<3|only@3>[Promise Fulfilled]
	\end{description}
\end{frame}

\begin{frame}{Promise States Relationships\cite{parker:2015}}
	\begin{tikzpicture}
		\node  at (0,2) [zero] (1){};
		\node  at (4,2) [vertex] (2){Pending};
		\node at (8,4)  [vertex] (3){Fulfilled};
		\node at (8,0)  [vertex] (4){Rejected};
		%edges
		\draw[Dedge] (1)  -> node[above]{Request\\Sweets} (2);
		\draw[Dedge] (2) -> node[above=5pt, rotate around={30:(0,0)}]{Homework\\Completed} (3);
		\draw[Dedge] (2) -> node[below=5pt, rotate around={-30:(0,0)}]{Homework\\Incomplete} (4);
		\node[right=of 3] (5){Response\\Sweets};
		\node[right=of 4] (4){Response\\Nothing};
	\end{tikzpicture}
\end{frame}

\begin{frame}[fragile]{Promise Syntax\cite{mead:2018}}%p370 mead
	\begin{columns}[T]
		\begin{column}{0.48\textwidth}
			\begin{itemize}
				\item<1-> Creation: object
				\item<2-> Anonymous Arrow Fn
				\item<3-> Asynchronous
				\item<4-> Resolve, Reject
			\end{itemize}
		\end{column}
		\begin{column}{0.48\textwidth}
			{\bf Listing}
			\begin{lstlisting}[language=JavaScript]
 var somePromise = new promise((resolve,reject)=>{
 //do asynchronous stuff here
	 });
			\end{lstlisting}
		\end{column}
	\end{columns}	
\end{frame}




\begin{frame}{Promise Example}{Resolve}
	\begin{columns}[T]
		\begin{column}{0.35\textwidth}
			\begin{block}{Output}
				{\tt success: It worked}
			\end{block}
		\end{column}
		\begin{column}{0.55\textwidth}
			{\bf Listing}
			\lstinputlisting[language=JavaScript]{promiseResolve.js}
		\end{column}
	\end{columns}	
\end{frame}

\begin{frame}{Promise Example}{Reject}
	\begin{columns}[T]
		\begin{column}{0.35\textwidth}
			\begin{block}{Output}
				{\tt Failure: It Failed}
			\end{block}
		\end{column}
		\begin{column}{0.55\textwidth}
			{\bf Listing}
			\lstinputlisting[language=JavaScript]{promiseReject.js}
		\end{column}
	\end{columns}	
\end{frame}


%notes: passing two functions, each with one parameter
%
\begin{frame}{Call a Promise}{.then}
	\begin{columns}[T]
		\begin{column}{0.35\textwidth}
			\begin{itemize}
				\item .then is a callback function 
					\begin{itemize}
						\item success
						\item failure
						\end{itemize}
				\item two callback functions
				\item Reject or Resolve
				\item Only reject once
				\item Only resolve once
				\item Pending for 500ms
			\end{itemize}
		\end{column}
		\begin{column}{0.55\textwidth}
			{\bf Listing}
			\lstinputlisting[language=JavaScript,firstline=8,lastline=13,firstnumber=8]{promiseReject.js}
		\end{column}
	\end{columns}	
\end{frame}

%
% p 378
\begin{frame}{Return a Promise}{Resolve}
	\begin{columns}[T]
		\begin{column}{0.2\textwidth}
			\begin{block}{Output}
				sum:109
			\end{block}
		\end{column}
		\begin{column}{0.75\textwidth}
			{\bf Listing}
			\lstinputlisting[language=JavaScript]{promiseReturnResolve.js}
		\end{column}
	\end{columns}	
\end{frame}

%
% p 378
\begin{frame}{Return a Promise}{Resolve}
	\begin{columns}[T]
		\begin{column}{0.20\textwidth}
			\begin{block}{Output}
				Error: enter two numbers	
			\end{block}
		\end{column}
		\begin{column}{0.75\textwidth}
			{\bf Listing}
			\lstinputlisting[language=JavaScript]{promiseReturnReject.js}
		\end{column}
	\end{columns}	
\end{frame}

% p 378
\begin{frame}{Promise Chaining \cite{parker:2015}}
	\begin{columns}[T]
		\begin{column}{0.20\textwidth}
			\begin{block}<2->{Output}
				ABC \\
				XYZ
			\end{block}
		\end{column}
		\begin{column}{0.75\textwidth}
			{\bf Listing}
			\lstinputlisting[language=JavaScript]{promiseChain.js}
		\end{column}
	\end{columns}	
\end{frame}


%
%\begin{frame}{}{}
%	\begin{columns}[T]
%	but in the reject example only reject is returned.
%		\end{column}
%		\begin{column}{0.48\textwidth}
%			{\bf Listing}
%			\lstinputlisting[language=JavaScript]{.js}
%		\end{column}
%	\end{columns}	
%\end{frame}



%\begin{frame}{}
%	\begin{columns}[T]
%		\begin{column}{0.48\textwidth}
%		\end{column}
%		\begin{column}{0.48\textwidth}
%			{\bf XXX}
%		\end{column}
%	\end{columns}	
%\end{frame}
%
%

%\begin{frame}{X}
%	\begin{columns}[T]
%		\begin{column}{0.48\textwidth}
%			\begin{itemize}
%				\item XXX 
%			\end{itemize}
%		\end{column}
%		\begin{column}{0.48\textwidth}
%			{\bf XXX}
%		\end{column}
%	\end{columns}	
%\end{frame}
\begin{frame}{Summary}
	\begin{itemize}
		\item Promises 
		\item passing functions as parameters
		\item Asynchronous code
		\item Synchronous code
		\item Promise chaining
		\item callback hell
		\item async, await, let
	\end{itemize}
\end{frame}


\begin{frame}[allowframebreaks]{References}
	\printbibliography
\end{frame}
	
\begin{frame}
	\frametitle{Web Resources}
	\begin{itemize}
	\item \url{http://hyperledger.org}
	\item \url{https://nodejs.org}
	\end{itemize}
\end{frame}

\end{document}
